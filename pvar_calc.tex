
% Filosofinis klausimas - gal galima irodineti lenteles forma?
% lentele nebutina, tiesiog tokia forma, kur aiskiai matosi lygmenys. 

\documentclass[12pt, a4paper]{article}

\usepackage[utf8]{inputenc}

\usepackage{lmodern}
\usepackage{layouts}                    % layouto vienetu transformacijos  
\usepackage[hyperref]{ntheorem}         % leidzia pakesti theoremu stiliu

\usepackage{amssymb}
\usepackage{amsmath}
\usepackage{listings}            % graziam kodui
\usepackage{enumerate}



% \usepackage{caption}                    % paveikleliu dejimas i veina vieta
% \usepackage{subcaption}                 % paveikleliu dejimas i veina vieta


% \usepackage{microtype}				          % isjungia triukus, kuriu nereikia
% \DisableLigatures[f]{encoding = *, family = * }

\usepackage{ifpdf}
\ifpdf
\usepackage{pdfpages}
\usepackage[pdftex]{hyperref}
\fi

\usepackage{color}
\definecolor{mygreen}{rgb}{0,0.6,0}
\definecolor{mygray}{rgb}{0.5,0.5,0.5}
\definecolor{FontGray}{rgb}{0.96,0.96,0.96}
\definecolor{mymauve}{rgb}{0.58,0,0.82}

\definecolor{tbred}{rgb}{1,0.6,0.6}
\definecolor{tbgreen}{rgb}{0.6,1,0.6}


\DeclareMathOperator*{\argmax}{arg\,max} 

\lstset{ %
  backgroundcolor=\color{FontGray},   % choose the background color; you must add \usepackage{color} or \usepackage{xcolor}
  basicstyle=\footnotesize,        % the size of the fonts that are used for the code
  breakatwhitespace=false,         % sets if automatic breaks should only happen at whitespace
  breaklines=true,                 % sets automatic line breaking
  captionpos=b,                    % sets the caption-position to bottom
  commentstyle=\color{mygreen},    % comment style
  deletekeywords={...},            % if you want to delete keywords from the given language
  escapeinside={\%*}{*)},          % if you want to add LaTeX within your code
  extendedchars=true,              % lets you use non-ASCII characters; for 8-bits encodings only, does not work with UTF-8
  frame=single,                    % adds a frame around the code
  keepspaces=true,                 % keeps spaces in text, useful for keeping indentation of code (possibly needs columns=flexible)
  keywordstyle=\color{blue},       % keyword style
  language=R,                      % the language of the code
  morekeywords={*,...},            % if you want to add more keywords to the set
  numbers=left,                    % where to put the line-numbers; possible values are (none, left, right)
  numbersep=5pt,                   % how far the line-numbers are from the code
  numberstyle=\tiny\color{mygray}, % the style that is used for the line-numbers
  rulecolor=\color{black},         % if not set, the frame-color may be changed on line-breaks within not-black text (e.g. comments (green here))
  showspaces=false,                % show spaces everywhere adding particular underscores; it overrides 'showstringspaces'
  showstringspaces=false,          % underline spaces within strings only
  showtabs=false,                  % show tabs within strings adding particular underscores
  stepnumber=1,                    % the step between two line-numbers. If it's 1, each line will be numbered
  stringstyle=\color{mymauve},     % string literal style
  tabsize=2,                       % sets default tabsize to 2 spaces
  title=\lstname                   % show the filename of files included with \lstinputlisting; also try caption instead of title
}
 
 
 
% stanratiniu fontu nustatymas
% \renewcommand{\sfdefault}{uhv}
% \renewcommand{\rmdefault}{utm}
% \renewcommand{\ttdefault}{ucr}


\theorembodyfont{\normalfont}
\theoremseparator{.}
\newtheorem{theorem}{Theorem}[section]
\newtheorem{lemma}[theorem]{Lemma}
\newtheorem{proposition}[theorem]{Proposition}
\newtheorem{definition}[theorem]{Definition}
\newtheorem{corollary}[theorem]{Corollary}


\newenvironment{proof}[1][Proof]{\noindent \textbf{#1.} }{\  \rule{0.5em}{0.5em}}
\numberwithin{equation}{section}

\title{The calculus of $p$-variation }



\begin{document}
  \maketitle


\begin{itemize}  
  \item $PP[a,b]$ --  the set of all point partitions of $[a,b]$
    (def.~\ref{def:pp}).
  \item $s_{p}(f;\kappa)$ --  $p$-variation sum (def.~\ref{def:pvar}).
  \item $v_{p}\left( f\right)$ -- $p$-variation of the function $f$
    (def.~\ref{def:pvar}).
  \item $SP_{p}(f,[a,b])$ -- the set of supreme partitions
    (def.~\ref{def:pvar}).
  \item $\overline{SP}_{p}(f,[a,b])$ -- a set of points that are
    in any supreme partition (def.~\ref{def:psp}).
  \item $PM[a,b]$ -- a set of piecewise monotone functions
    (def.~\ref{def:PM}).  
  \item $CPM[a,b]$ -- a set of continuous piecewise monotone functions
    (def.~\ref{def:PM}).
  \item $K(f,[a,b])$ -- minimal size of PM partitions
    (prop.~\ref{prop:K_f}).
  \item $X(f,[a,b])$ -- the set of PM partitions with minimal size
    (def.~\ref{def:X_f}).
\end{itemize}   

%Turning points
%lap - the monotonic interval
%corner
%local extrema
%forward (backward) maximum (minimum) (extrema) 

\section{Mathematical analysis}

\subsection{General known properties}
  
  
%%% Definition of partition %%% 
% Status
\begin{definition}[Partition]\label{def:pp}
  Let $J = [a,b]$ be a closed interval of real numbers with 
  $-\infty < a \leq b <+\infty$. 
  If $a < b$, an ordered set $\kappa = \{x_{i}\}_{i=0}^{n}$ 
  of points in $[a,b]$ such that a  
  $a=x_{0}<x_{1}<x_{2}<...<x_{n}=b$ is called a \emph{(point) partition}. 
  The size of the partition is denoted $\left|\kappa \right| :=\#\kappa-1=n$. 
  The set of all point partitions of $[a,b]$ is denoted by $PP[a,b]$.  
\end{definition} 
 
%%% Definition of $p$-variation %%% 
% Status:OK
\begin{definition}[$p$-variation]\label{def:pvar}
  Let $f:[a,b] \rightarrow \mathbb{R}$ be a real function from 
  an interval $[a,b]$.
  If $a<b$, for  $\kappa =\{x_{i}\}_{i=0}^{n} \in PP[a,b]$ 
  the \emph{$p$-variation sum} is 
  \begin{equation}\label{eq:def_pvarsum}
    s_{p}(f,\kappa):=\sum_{i=1}^{n}| f( x_{i}) -f( x_{i-1})|^{p},
  \end{equation}
  where $0<p<\infty $. Thus, the \emph{$p$-variation} 
  of $f$ over $[a,b]$ is $0$ if $a = b$ and otherwise
  \begin{equation}\label{eq:def_pvar}
    v_{p}(f) =v_{p}(f,[a,b]) :=\sup \left\{
    s_{p}(f,\kappa ):\kappa \in PP[a,b]\right\}.    
  \end{equation}
  
  The partition $\kappa$ is called \emph{supreme partition} 
  if it satisfies the property $v_{p}(f) = s_{p}(f,\kappa)$.
  The set of such partitions is denoted
  $SP_{p}(f,[a,b])$.
  
\end{definition}  
 
Vidinis Komentaras (VK): Dar neaisku kaip pakrikstyti skaidini, kuris pasiekia supremuma. 
 
 
 
%%% THe properties of $p$-variation %%% 
% Status:OK 
\begin{lemma}[Elementary properties]\label{lm:element_properties} 
  Let $f:[a,b] \rightarrow \mathbb{R}$ and $0<p<\infty $. 
  Then the following $p$-variation properties holds
  \begin{enumerate}[a)]
    \item \label{lm:ep_a} $ v_p(f,[a,b]) \geq 0 $,
    \item \label{lm:ep_b} $ v_p(f,[a,b]) = 0 \Leftrightarrow f \equiv Const. $,
    \item \label{lm:ep_c} $ \forall C \in \mathbb{R}: v_p(f+C,[a,b]) = v_p(f,[a,b]) $,  
    \item \label{lm:ep_d} $ \forall C \in \mathbb{R}: v_p(Cf,[a,b]) = C^p v_p(f,[a,b]) $,  
    \item \label{lm:ep_e} $ \forall c \in [a,b]: v_p(f,[a,b]) \geq v_p(f,[a,c]) + v_p(f,[c,b]) $,  
    \item \label{lm:ep_f} $ \forall [a',b'] \subset [a,b] : v_p(f,[a,b]) \geq v_p(f,[a',b']) $.  
    \item \label{lm:ep_g} $ \forall \kappa \in PP[a,b]:s_{p}(f;\kappa) \leq v_p(f,[a,b]) $.  
  \end{enumerate}

  All listed properties are elementary derived directly form the $p$-variation definition.
\end{lemma}
 
 

%%% Regulated function (definition) %%% 
% Status:OK  
\begin{definition}[Regulated function](\cite{Qian}, Def. 3.1)
  For any interval $J$, which may be open or closed at either end, 
  real function $f$ is called \emph{regulated} on $J$ 
  if it has left and right limits   
  $f(x-)$ and $f(x+)$ respectively at each point $x$ in interior of $J$,
  a right limit at the left end point and a left limit 
  at the right endpoint.
\end{definition}  
 
%%% Limit points %%% 
% Status:OK   
\begin{proposition}(\cite{Qian}, Lemma 3.1)\label{prop:LimPoints} 
  Let $1 \leq p < \infty$. If $f$ is regulated then $v_{p}(f)$ remains the same
  if points $x+$, $x-$ are allowed as partition points $x_i$ in the definition \ref{def:pvar}.
\end{proposition}  
 
%%% Piecewise monotone functions (definition) %%% 
% Status:OK    
\begin{definition}[Piecewise monotone functions](\cite{Qian}, Def. 3.2)\label{def:PM}
  A regulated real-valued function $f$ on closed interval $[a,b]$ 
  will be called \emph{piecewise monotone} (PM)
  if there are points $a=x_0<\dots<x_k=b$ for some finite $k$ such 
  that $f$ is monotone on each interval $[x_{j-1},x_j],\;j=1,\dots,k$. 
  Here for $j=1,\dots,k-1$, $x_j$ may be point $x-$ or $x+$.
  The set of all piecewise monotone functions is denoted $PM=PM[a,b]$.
  
  In addition to PM, if $f$ is continuous function we will call it 
  continuous piecewise monotone(CPM). 
  The set of such functions is denoted $CPM=CPM[a,b]$. 
\end{definition}   


 
%%% Minimal size of the partitions %%% 
% Status:OK    
\begin{proposition}(\cite{Qian}, Prop. 3.1)\label{prop:K_f}
  If $f$ is PM, there is a minimal size of partition $|\kappa|$ for which the definition \ref{def:PM} holds.
  The minimal size of the the PM partition is denoted $K(f,[a,b])=K(f)$, namely
  \begin{equation}
  K(f):=\min\left\{n:\exists \{x_{i}\}_{i=0}^{n} \in PP[a,b]:  f \text{ is monotonic in each } [x_{j-1}, x_j]   \right\}.
  \end{equation}
\end{proposition}    
 
%%% The set of minimal partitions %%% 
% Status:OK  
\begin{definition}[The set of PM partitions with minimal size](\cite{Qian}, Def. 3.3)\label{def:X_f}
  If $f$ is PM, let $X(f)=X(f,[a,b])$ be the set of all $\{x_{i}\}_{i=0}^{K(f)}$ for which the definition 
  of PM (def.~\ref{def:PM}) holds. $X(f)$ is called the \emph{set of PM partitions with minimal size}.
\end{definition}    
 
%%% uniqueness of difference of PM partition %%%
% Status:OK  
\begin{proposition}(\cite{Qian}, Prop. 3.3)\label{prop:PM_Unique}
  Let $f$ is PM then the numbers 
  $\alpha_j(f):=f(x_j) - f(x_{j-1})$ for
  $\{ x_j \}_{j=0}^{K(f)} \in X(f)$ and 
  $j=1,2,\dots K(f)$
  are uniquely determined.
\end{proposition}  
 
 
%%% PM form: alternating laps %%%
% Status: RECHECK  with orginal.
\begin{proposition}(\cite{Qian}, Prop. 3.2)\label{prop:PM_form}
  Let $f$ is PM. For any partition 
  $\{ x_j \}_{j=0}^{K(f)} \in X(f)$ 
  exactly one of the flowing stamens holds:
  \begin{enumerate}[(a)]
   \item $f(x_0)>f(x_1)<f(x_2)>\dots$. 
     Function $f$ is not
     increasing in intervals $[x_{2j}, x_{2j+1}]$, 
     then $2j+1\leq K(f)$. 
     Function $f$ is not
     decreasing in intervals $[x_{2j-1}, x_{2j}]$, 
     then $j \geq 1$ and $2j\leq K(f)$.  
     
   \item (a) holds for a function $-f$; or
   \item $K(f, [a,b]) = 1$ and $f$ is constant in interval $[a,b]$.
  \end{enumerate}
  
  
\end{proposition}  
  
 
 
%%% Equality by PM %%% 
% Status:OK 
\begin{definition}[The equality by PM](\cite{Qian}, Def. 3.4)\label{def:EqPM}
  If $f$, $g$ are two PM functions, possibly on different intervals, such that 
  $K(f)=K(g)$ and $\alpha_j(f)=\alpha_j(g)$ for $j=1,2,\dots,K(f)$, then we say that
  $f \stackrel{PM}{=} g$.
\end{definition}    
  
%%% f {DM}= g %%%
% Status:OK 
\begin{proposition}(\cite{Qian}, Cor. 3.1)\label{prop:f_PM_g}
  Let $p>1$ and functions $f$ and $g$ are PM. 
  If $f \stackrel{PM}{=} g$ or $f \stackrel{PM}{=} -g$, then $v_p(f)=v_p(g)$.
\end{proposition} 

%%% Exists meaningful partition in set of PM partition. %%%
% Status:OK 
\begin{proposition}(\cite{Qian}, Them. 3.1)\label{prop:sup_in_PM}
  Let $f$ is PM, $\kappa \in X(f)$ and $1 \leq p < \infty$. 
  Then the supremum of $p$-variation in Definition~\ref{def:pvar}
  is attained for some partition $r \subset  \kappa$.
\end{proposition} 
\begin{corollary}\label{cor:SPNotEmpty}
  The set $SP_{p}(f,[a,b])$ is not empty for all $f \in PM[a,b]$.  
\end{corollary}


%%% Sample function (definition) %%%
% Status:OK 
\begin{definition}[Sample function]\label{def:Seq2Fun}
  Suppose $X=\{X_{i}\}_{i=0}^{n}$ is any sequence real numbers. 
  We will call such sequence a \emph{sample}, 
  whereas $n$ will be referred to as a \emph{sample size}.
  Then the \emph{sample function}  
  $G_X:[0,n] \rightarrow \mathbb{R}$ is defined as 
  \begin{eqnarray}
    G_X(t) := X_{\lfloor t \rfloor},\;t\in[0,n],
  \end{eqnarray}
  where $\lfloor t \rfloor$ denotes floor function at point $t$. 
\end{definition} 

%%% $p$-variation of the sequence (definition) %%%
% Status:OK 
\begin{definition}[$p$-variation of the sequence]\label{def:pvarseq}
  Let $X=\{X_i\}_{i=0}^n$. The $p$-variation
  of the sample $X$ is defined as $p$-variation of the 
  function $G_X(t)$, namely
  \begin{equation}
    v_p(X) := v_p(G_X(t),[0,n]).
  \end{equation}
\end{definition}


%%%%%%%%%%%%%%%%%%%%%%%%%%%%%%%%%%%%%%%%%%%%%%%%%%%%%%%%%%%%%%%%%%%%%%%
\subsection{General properties with proofs}


%%% VIP: p-variation is additive in meaningful points %%%
\begin{proposition}\label{prop:pvar_sum_x} 
  Let $f:[a,b] \rightarrow \mathbb{R}$ be PM and
  $\{x_i\}_{i=0}^n \in PP[a,b]$ is any partition of interval $[a,b]$. 
  Then the statement 
  \begin{equation}
    \exists \kappa \in SP_{p}(f,[a,b]):\forall i,x_i \in \kappa
  \end{equation}
  is equivalent to 
  \begin{equation}
    v_p(f,[a,b]) = \sum_{i=1}^n v_p(f,[x_{i-1},x_i]).
  \end{equation}
\end{proposition}
\begin{proof}
  Necessary. Let $f:[a,b] \rightarrow \mathbb{R}$ be PM,
  $\{x_i\}_{i=0}^n \in PP[a,b]$ and 
  \begin{equation}
    \exists \kappa \in SP_{p}(f,[a,b]):\forall i,x_i \in \kappa. 
  \end{equation}
  
  Points from the partition $\kappa$ will be denoted $t_i$, 
  i.e. $\kappa=\{t_i\}_{i=0}^m$.
  Then, according to definitions of $SP_{p}$ and 
  $p$-variation (def.~\ref{def:pvar}) the following equation holds
  \begin{equation}\label{eq:ir1.eq1}
  v_p(f,[a,b])=s_{p}(f;\kappa)=\sum_{j=1}^{m}|f(t_{j-1})-f(t_j)|^p 
    = \sum_{i=1}^n\sum_{j=h(i-1)+1}^{h(i)}|f(t_{j-1})-f(t_j)|^p,
  \end{equation}  
  where $h:\{0,1,\dots,n\} \rightarrow \{0,1,\dots,m\}$ 
  denotes a function 
  from the set of index of $x$ to the set of index of $t$, namely:
  \begin{equation}\label{eq:ir1.eqx1}
    h(i):=(j_i : x_i = t_{j_i}=t_{h(i)}).
  \end{equation}  
  The equation (\ref{eq:ir1.eq1}) holds, because all the elements 
  in the sum remains, we just grouped them.
  
  Moreover, the inequality  
  \begin{equation}\label{eq:ir1.eq2}
  \sum_{j=h(i-1)+1}^{h(i)}|f(t_{j-1})-f(t_j)|^p 
    \leq v_p(f,[x_{i-1},x_i])
  \end{equation}  
  holds according to Lemma \ref{lm:element_properties}(\ref{lm:ep_g}).
  
  As a result of (\ref{eq:ir1.eq1}) and (\ref{eq:ir1.eq2}) we get
  \begin{equation}\label{eq:ir1.eq3}
    v_p(f,[a,b]) \leq \sum_{i=1}^n v_p(f,[x_{i-1},x_i]).
  \end{equation}
  
  On the other hand, according to the same
  Lemma~\ref{lm:element_properties}(\ref{lm:ep_e}) 
  the following inequality holds
  \begin{equation}\label{eq:ir1.eq4}
    v_p(f,[a,b]) \geq \sum_{i=1}^n v_p(f,[x_{i-1},x_i]).
  \end{equation}
  
  Finally, from the (\ref{eq:ir1.eq3}) and (\ref{eq:ir1.eq4}) follows 
  \begin{equation}\label{eq:ir1.eq5}
    v_p(f,[a,b]) = \sum_{i=1}^n v_p(f,[x_{i-1},x_i]).
  \end{equation}
  
  Sufficiency. Suppose $f:[a,b] \rightarrow \mathbb{R}$ and 
  \begin{equation}\label{eq:ir1.eq6}
    v_p(f,[a,b]) = \sum_{i=1}^n v_p(f,[x_{i-1},x_i]).
  \end{equation}
  
  According to Corollary~\ref{cor:SPNotEmpty}, 
  sets $SP_{p}(f,[x_{i-1},x_i]),\;i=1,\dots,n$ are not empty.  
  Lets take any partition from each of the 
  sets $SP_{p}(f,[x_{i-1},x_i]),\;i=1,\dots,n$ and denote it $\kappa_i$.
    
  
  Then, let define a joint partition $\kappa:=\cup_{i=1}^n \kappa_i$. 
  Points from the partition $\kappa$ will be denoted by $t_i$. 
  In addition, we will use the function $h$, 
  which is defined in (\ref{eq:ir1.eqx1}).
  Then, continuing the equation (\ref{eq:ir1.eq6}) we get
  \begin{equation}\label{eq:ir1.eq7}
  v_p(f,[a,b]) = 
    \sum_{i=1}^n\sum_{j=h(i-1)+1}^{h(i)}|f(t_{j-1})-f(t_j)|^p
    =\sum_{i=1}^{|\kappa|}|f(t_{i-1})-f(t_i)|^p.
  \end{equation}
  This means that $\kappa \in MP_{p}(f,[a,b])$. 
  Moreover, $\forall i:x_i \in \kappa$,
  because $\kappa=\cup_{i=1}^n \kappa_i$.  
\end{proof}


% The subset on the points of supreme partition is aditive
\begin{lemma}\label{lm:pvarSubset}
  Let $f:[a,b] \rightarrow \mathbb{R}$ be PM 
  and $\{x_i\}_{i=0}^{n} \subset \kappa \in SP_{p}(f,[a,b])$. Then
  \begin{equation}
    \forall k,l: v_p(f,[x_k,x_l]) = 
      \sum_{i=k+1}^l v_p(f,[x_{i-1},x_i]),\; 0 \leq k<l\leq n. 
  \end{equation}
\end{lemma}
\begin{proof}
  Suppose $\{x_i\}_{i=0}^{n} \subset \kappa \in SP_{p}(f,[a,b])$. 
  Let choose $k$ and $l$ such that $0\leq k<l\leq n$.
  Lets apply Proposition~\ref{prop:pvar_sum_x} for the partition
  $\{x_0,x_k,x_{k+1},\dots,x_{l-1},x_l,x_n\}$. Thus,
  \begin{equation}
    v_p(f,[a,b]) = v_p(f,[x_{0},x_k]) 
      + \sum_{i=k+1}^l v_p(f,[x_{i-1},x_i]) + v_p(f,[x_l,x_n]). 
  \end{equation}
  In addition, we can apply the same proposition for 
  the partition $\{x_0,x_k,x_l,x_n\}$, then
  \begin{equation}
    v_p(f,[a,b]) =  v_p(f,[x_{0},x_k]) + v_p(f,[x_k,x_l]) 
      + v_p(f,[x_l,x_n]). 
  \end{equation}
  By subtracting one equation form the other we get the result that
  \begin{equation}
    v_p(f,[x_k,x_l])= \sum_{i=k+1}^l v_p(f,[x_{i-1},x_i]). 
  \end{equation}
\end{proof}

\begin{lemma}\label{lm:spSubset}
  Let $f:[a,b] \rightarrow \mathbb{R}$ be PM 
  and $\{t_i\}_{i=0}^{n} \in SP_{p}(f,[a,b])$. Then
  \begin{equation}
    \forall k,l: v_p(f,[t_k, t_l]) = 
      s_p\left(f,\{t_i\}_{i=k}^{l}\right) ,\; 0 \leq k<l\leq n. 
  \end{equation}
\end{lemma}
\begin{proof}
  Let $f:[a,b] \rightarrow \mathbb{R}$ be PM 
  and $\{t_i\}_{i=0}^{n} \in SP_{p}(f,[a,b])$. 
  Let choose $k$ and $l$ such that $0\leq k<l\leq n$.
  Then 
  \begin{eqnarray}
    v_p(f, [a, b]) & = & s_p(f,\{t_i\}_{i=0}^{n}) \\
    & = & s_p(f,\{t_i\}_{i=0}^{k})+s_p(f,\{t_i\}_{i=k}^{l})
      +s_p(f,\{t_i\}_{i=l}^{n})\\
    \label{eq:spSubset_p1}  
    & \leq & v_p(f, [a,t_k]) + s_p(f,\{t_i\}_{i=k}^{l})
      +v_p(f, [t_l, b]).         
  \end{eqnarray}    
  The last inequality holds according to 
  Lemma~\ref{lm:element_properties}(\ref{lm:ep_g}).  
    
  On the other hand, from Proposition~\ref{lm:pvarSubset}
  we get
  \begin{equation}\label{eq:spSubset_p2}  
    v_p(f, [a, b]) = 
      v_p(f, [a,t_k]) + v_p(f,[t_k, t_l])
      +v_p(f, [t_l, b]).     
  \end{equation}      
  
  Form (\ref{eq:spSubset_p1}) and (\ref{eq:spSubset_p2})
  follows 
  \begin{equation} 
    v_p(f,[t_k, t_l]) \leq s_p(f,\{t_i\}_{i=k}^{l}).  
  \end{equation} 
  
  Finally, from Lemma~\ref{lm:element_properties}(\ref{lm:ep_g})
  we conclude that 
  \begin{equation} 
    v_p(f,[t_k, t_l]) = s_p(f,\{t_i\}_{i=k}^{l}).  
  \end{equation}   
  
\end{proof}


% The point of supreme partition 
\begin{definition}[The point of supreme partition]\label{def:psp}
  Let $f:[a,b] \rightarrow \mathbb{R}$ be PM.  
  The point $x$ will be called the \emph{point of supreme partition}
  if 
  \begin{equation}
    \exists \kappa \in SP_p(f, [a,b]): x \in \kappa.
  \end{equation} 
  The set of such points will be denoted by 
  $\overline{SP}_p(f, [a, b])$.
\end{definition}


\begin{lemma}\label{lm:psp} 
  Suppose $f:[a,b] \rightarrow \mathbb{R}$ is PM, 
  $x \in [a,b]$,
  $x \notin \overline{SP}_p(f, [a, b])$ and
  $\{t_i\}_{i=0}^n \in SP(f, [a, b])$ is any supreme partition.
  Then, 
  \begin{equation}
    \exists j=1,\dots,n: x \in (t_{j-1}, t_j) \text{ and } 
    x \notin \overline{SP}_p(f, [t_{j-1}, t_j]).
  \end{equation}
\end{lemma}
\begin{proof}
  Suppose the assumptions of lemma is valid.
  Since $x \in [a,b]$ and $[a, b] = \cup_{i=1}^{n}[t_{i-1}, t_{i}]$,
  then 
  \begin{equation}\label{eq:psp_p1}
    \exists j=1,\dots,n:x \in [t_{j-1}, t_{j}].
  \end{equation}    
  Moreover, $x \notin \{t_i\}_{i=0}^n$, 
  because $x \notin \overline{SP}_p(f, [a, b])$, thus,
  $x \neq t_{j-1}$ and  $x \neq t_{j}$. 
  In addition to (\ref{eq:psp_p1}) this means that 
  $x \in (t_{j-1}, t_{j})$.
    
  Now, we will proof that 
  $x \notin \overline{SP}_p(f, [t_{j-1}, t_j])$.
  Suppose to the contrary that 
  $x \in \overline{SP}_p(f, [t_{j-1}, t_j])$. 
  Then, according to definition of 
  $\overline{SP}_p(f, [t_{j-1}, t_j])$, 
  \begin{equation}
    \exists \kappa \in SP_p(f, [t_{j-1}, t_j]): x \in \kappa.
  \end{equation}    
  Since, $\kappa \in SP_p(f, [t_{j-1}, t_j])$, 
  then 
  \begin{equation}
    v_p(f, [t_{j-1}, t_j])=s_p(f, \kappa)
  \end{equation} 
  
  Applying Proposition~\ref{prop:pvar_sum_x} 
  for partition $\{t_i\}_{i=0}^n$ we get
  \begin{eqnarray}
    v_p(f, [a, b]) &=& v_p(f, [t_{0}, t_{j-1}]) 
     + v_p(f, [t_{j-1}, t_j]) + v_p(f, [t_{j}, t_n])\\
     &=& s_p(f, \{t_i\}_{i=0}^{j-1}) 
     + s_p(f, \kappa) + s_p(f, \{t_i\}_{i=j}^{n}) 
  \end{eqnarray}
   
  This means that the partition 
  $r:=\{t_i\}_{i=0}^{j-1} \cup \kappa \cup \{t_i\}_{i=j}^{n}$
  is supreme partition, so 
  $x \in r \in SP(f, [a, b])$, therefore, by definition
  $x \in \overline{SP}_p(f, [a, b])$. 
  This contradict to initial assumption. 
\end{proof}


\begin{definition}[$f$-join]\label{def:fjoin}
  Suppose $f:[a,b] \rightarrow \mathbb{R}$ is PM.
  We will say that points $t_a$ and $t_b$ ($t_a<t_b$) 
  are \emph{$f$-joined} in interval $[a,b]$ if 
  \begin{equation}
    \exists \{x_j\}_{j=0}^{n} \in SP_{p}(f,[a,b]):
      [t_a,t_b]=[x_{j-1},x_j], \text{ with some } j.
  \end{equation}  
\end{definition}

% suziureti zenklus 
\begin{lemma}\label{lm:fjoin} 
  Suppose $f:[a,b] \rightarrow \mathbb{R}$ is PM and 
  points $t_a$ and $t_b$ ($t_a<t_b$) are 
  $f$-jointed in interval $[a,b]$.
  Then all following statements holds   
  \begin{enumerate}[a)]
    \item \label{lm:fjoin_a} 
    $v_p(f, [t_a,t_b] = \left|f(t_a)-f(t_b)\right|^p$;
    
    \item \label{lm:fjoin_b} 
    Let $x \in [t_a, t_b]$. 
    If $f(t_a) \geq f(t_b)$, then $f(t_a) \geq f(x) \geq f(t_b)$.
    If $f(t_a) \leq f(t_b)$, then $f(t_a) \leq f(x) \leq f(t_b)$;
    
  \end{enumerate}
\end{lemma}
\begin{proof}
  \paragraph*{a)}
  Let $f:[a,b] \rightarrow \mathbb{R}$  be PM, 
  $t_a<t_b$ and pair of points $t_a$, $t_b$ are $f$-joined. 
  Then exists $\{x_j\}_{j=0}^{n}$ and $j$ from the
  Definition~\ref{def:fjoin}.
  Thus, according to Lemma~\ref{lm:spSubset}
  \begin{equation}\label{eq:lm:fjoin_a}
    v_p(f,[t_{j-1}, t_j]) = 
      s_p\left(f,\{t_{j-1}, t_j\}\right) = \left|f(t_a)-f(t_b)\right|^p.
  \end{equation}      
  
  \paragraph*{b)}
  Let $f:[a,b] \rightarrow \mathbb{R}$ be PM 
  and points $t_a$ and $t_b$ are $f$-joined.
  Since $f \stackrel{PM}{=} -f$, with out loss of 
  generality we can assume that
  $f(t_a) \leq f(t_b)$.  
  Suppose to the contrary that $f(t_b)$ 
  is not an extrema of the function in interval $[t_a,t_b]$.
  Hence, $\exists c \in [t_a, t_b]: f(c)>f(t_b)$. Therefore,
  $|f(c)-f(t_a)|^p>|f(t_b)-f(t_a)|^p$.
  According to (\ref{eq:lm:fjoin_a}), 
  $v_p(f,[t_a,t_b])=|f(t_b)-f(t_a)|^p$, 
  thus, $|f(c)-f(t_a)|^p>v_p(f,[t_a,t_b])$,
  but this contradicts the definition of $p$-variation.  
  So, point $t_b$ must be an extrema in interval $[t_a,t_b]$.
  Symmetric arguments could be used for point $t_a$.  
  
  
\end{proof}




%%% The monotonisy of Sp function %%%
\begin{lemma}\label{prop:Sp_Monoton}
  Suppose $C \in \mathbb{R}$, 
  $c_1\geq0$ and $1\leq p<\infty$. 
  Then function
  $f:[0,\infty) \rightarrow \mathbb{R}$ with the values
  \begin{equation}
  f(x) = ( x + c_1  )^p - x^p - C,\;x \in [0,\infty),  
  \end{equation}
  are non decreasing in interval $[0,\infty)$. 
\end{lemma}
\begin{proof}
  Suppose $C \in \mathbb{R}$, $c_1\geq0$ and $1\leq p<\infty$. 
  Then, for all $x \geq 0$, the derivative of the function $f$ is
  \begin{eqnarray*} 
    f'(x)   &=& p(x+c_1)^{p-1} - px^{p-1} \\
          &\geq& px^{p-1} - px^{p-1} = 0.
  \end{eqnarray*}  
  The derivative of function $f$ is non negative, 
  thus the function $f$ is non decreasing, if $x \geq 0$.
\end{proof}
\begin{corollary}\label{cor:convex}
  Suppose $c_1\geq0$, $C \in \mathbb{R}$, $1\leq p<\infty$ 
  and $0 \leq x \leq y$. 
  Then the following implication holds 
  \begin{equation}
  |x + c_1 |^p > x^p + C \Rightarrow |y + c_1 |^{p} > y^{p} + C.  
  \end{equation}
\end{corollary}
\begin{proof}
  Suppose  $0 \leq x \leq y$. 
  Since $f$ is non decreasing, $f(x) \leq f(y)$.
  Therefore, if $f(x)>0$, then $f(y)>0$.
\end{proof}


%%% VIP: if pont is insignificant, then it will remain insignificant %%%
\begin{proposition}\label{prop:notinST} 
  Suppose $f:[a,b] \rightarrow \mathbb{R}$ is PM and
  $x \in [a', b'] \subset [a,b]$.    
  If $x \notin \overline{SP}_p(f, [a', b'])$, then 
  $x \notin \overline{SP}_p(f, [a, b])$. 
\end{proposition}

\begin{proof}
  Let $f:[a,b] \rightarrow \mathbb{R}$ be PM,
  $x \in [a', b'] \subset [a,b]$, 
  and $ x \notin \overline{SP}_p(f, [a', b'])$.
  Suppose to the contrary that
  $x \in \overline{SP}_p(f, [a, b])$.
  
  Since $x \in \overline{SP}_p(f, [a, b])$,
  according to the Definition~\ref{def:psp},
  \begin{equation} \label{eq:notinST_p1} 
    \exists \{t_i\}_{i=0}^n \in SP_p(f, [a, b]): x \in \{t_i\}_{i=0}^n.
  \end{equation}     
  
  Let $\{y_i\}_{i=0}^n \in SP_p(f, [a', b'])$ be any 
  supreme partition from the interval $[a', b']$. 
  Then, according to Lemma~\ref{lm:psp}, 
  \begin{equation} \label{eq:notinST_p2} 
    \exists j=1,\dots,n: x \in (y_{j-1}, y_j) \text{ and } 
    x \notin \overline{SP}_p(f, [y_{j-1}, y_j]).
  \end{equation}     
  Moreover, 
  \begin{equation}\label{eq:notinST_p3} 
    x \notin \{a, y_{j-1}, y_j, b\},
  \end{equation} 
  because $x \in (y_{j-1}, y_j) \subset [a, b]$.  
  Thus, from (\ref{eq:notinST_p1}) and (\ref{eq:notinST_p3}) 
  follows that  
  \begin{equation}\label{eq:notinST_p4}
    \exists r \in \{1,2,\dots,n-1\} : t_r = x.
  \end{equation}  

  Lets denote variables $l$ and $k$ by
  \begin{eqnarray*}
    l &:=& \max \left\{i \in \{1,2,\dots,n-1\} : 
      t_i  \in (y_{j-1}, y_j) \right\}, \\
    k &:= &\min \left\{i \in \{1,2,\dots,n-1\} : 
      t_i  \in (y_{j-1}, y_j) \right\}.
  \end{eqnarray*}  
  Since $x \in (y_{j-1}, y_j)$ and (\ref{eq:notinST_p4})~holds, 
  the values $l$ and $k$ always exists and $k \leq r \leq l$. 
  According to $l$ and $k$ definitions the following inequality holds  
  \begin{equation}\label{eq:notinST_p5}
    t_{k-1} \leq y_{j-1} < t_{k} \leq x \leq t_{l} < y_j \leq t_{l+1}.
  \end{equation}
  
 
  According to Lemma~\ref{lm:element_properties}(\ref{lm:ep_e})
  \begin{equation}\label{eq:notinST_p6}
    v_p(f,[y_{j-1},y_j]) \geq v_p(f,[y_{j-1},t_{k}]) 
      + v_p(f,[t_{k},t_{l}]) + v_p(f,[t_{l},y_j]),
  \end{equation}      
  Firstly, let suppose   
  \begin{equation}\label{eq:notinST_p7}
    v_p(f,[y_{j-1},y_j]) = v_p(f,[y_{j-1},t_{k}]) 
      + v_p(f,[t_{k},t_{l}]) + v_p(f,[t_{l},y_j]).
  \end{equation}   
  Lets take any supreme partitions from intervals
  $[y_{j-1},t_{k}]$ and  $[t_{l},y_j]$, namely
  $\kappa_k \in SP_p(f,[y_{j-1},t_{k}]$ and
  $\kappa_l \in SP_p(f,[t_{l},y_j]$.  
  In addition, since $\{t_i\}_{i=0}^n \in SP_p(f, [a, b])$,
  according to Lemma~\ref{lm:spSubset}
  $v_p(f,[t_{k},t_{l}]) = s_p(f,\{t_i\}_{i=k}^l)$.
  Therefore,
  \begin{equation*}\label{eq:notinST_p8}
    v_p(f,[y_{j-1},y_j]) = s_p(f,\kappa_k) 
      + s_p(f,\{t_i\}_{i=k}^l) + v_p(f,\kappa_l).
  \end{equation*}  
  This means, that 
  \begin{equation*}\label{eq:notinST_p8}
    \kappa_k \cup \{t_i\}_{i=k}^l \cup \kappa_l 
      \in SP_p(f, [y_{j-1}, y_j].
  \end{equation*} 
  So, 
  $x \in \overline{SP}_p(f, [y_{j-1}, y_j])$, 
  because $x \in \{t_i\}_{i=k}^l$.
  This contradicts (\ref{eq:notinST_p2}),
  thus, equality (\ref{eq:notinST_p7}) is not valid.
  As a result, inequality (\ref{eq:notinST_p6}) becomes
  \begin{equation}\label{eq:notinST_p9}
    v_p(f,[y_{j-1},y_j]) > v_p(f,[y_{j-1},t_{k}]) 
      + v_p(f,[t_{k},t_{l}]) + v_p(f,[t_{l},y_j]).
  \end{equation}    
  
  Since points $y_{j-1}$ and $y_j$ are $f$-joined,
  from the Lemma~\ref{lm:fjoin}(\ref{lm:fjoin_a}) we get
  \begin{equation*}\label{eq:notinST_p10}
    v_p(f,[y_{j-1},y_j]) = |f(y_{j-1})-f(y_j)|^p,
  \end{equation*}   
  therefore,
  \begin{equation*}\label{eq:notinST_p11}
    |f(y_{j-1})-f(y_j)|^p > v_p(f,[y_{j-1},t_{k}]) 
      + v_p(f,[t_{k},t_{l}]) + v_p(f,[t_{l},y_j]). 
  \end{equation*}        
  Moreover, according to 
  Lemma~\ref{lm:element_properties}(\ref{lm:ep_g}), 
  from the last statement follows
  \begin{equation} \label{eq:notinST_p12}
    |f(y_{j-1})-f(y_j)|^p > |f(y_{j-1})-f(t_{k})|^p +
       v_p(f,[t_k,t_l]) +  |f(t_{l})-f(y_j)|^p.   
  \end{equation}  
  
  Since $v_p(f)=v_p(-f)$ (see Proposition~\ref{prop:f_PM_g}), 
  with out loss of generality we can assume that
  $f(y_{j-1}) \geq f(y_j)$. 
  Hence, from the Lemma~\ref{lm:fjoin}(\ref{lm:fjoin_b}) 
  we get that
  \begin{eqnarray}
    f(y_{j-1}) \geq f(t_{l}) \geq f(y_j), \label{eq:notinST_13}  \\
    f(y_{j-1}) \geq f(t_{k}) \geq f(y_j). \label{eq:notinST_14}
  \end{eqnarray}  

  In addition, the pairs of points $\{t_l,t_{l+1}\}$ 
  and $\{t_{k-1},t_{k}\}$ are also $f$-joined, therefore,
  by the Lemma~\ref{lm:fjoin}(\ref{lm:fjoin_b})
  inequalities (\ref{eq:notinST_13}) and (\ref{eq:notinST_14})
  could be extended as
  \begin{eqnarray}
    \label{eq:notinST_15} 
    f(y_{j-1}) \geq f(t_{l}) \geq f(y_j) \geq f(t_{l+1}),  \\
    \label{eq:notinST_16}
    f(t_{k-1}) \geq f(y_{j-1}) \geq f(t_{k}) \geq f(y_j). 
  \end{eqnarray}         
    
  
  From the (\ref{eq:notinST_15}) the following inequalities holds
  \begin{eqnarray*}
    f(y_{j-1})-f(t_l) &\geq& 0, \\
    f(t_{l})-f(t_{l+1}) \geq f(t_l)-f(y_j) & \geq & 0.
  \end{eqnarray*} 
  Therefore, we can use Corollary~\ref{cor:convex}. According to it,
  from the inequality (\ref{eq:notinST_p12}) follows  
  \begin{eqnarray*}
    |f(y_{j-1})-f(t_l)+f(t_l)-f(y_j)|^p &>& |f(y_{j-1})-f(t_{k})|^p +
       v_p(f,[t_k,t_l]) +  |f(t_{l})-f(y_j)|^p, \\  
    |f(y_{j-1})-f(t_l)+f(t_l)-f(t_{l+1})|^p &>&|f(y_{j-1})-f(t_{k})|^p
      + v_p(f,[t_k,t_l]) +  |f(t_{l})-f(t_{l+1})|^p, \\
    |f(y_{j-1})-f(t_{l+1})|^p &>& |f(y_{j-1})-f(t_{k})|^p 
      + v_p(f,[t_k,t_l]) +  v_p(f,[t_l,t_{l+1}]). 
  \end{eqnarray*}    
  The last inequality holds,
  because points $t_{l}$ and $t_{l+1}$ are $f$-joined,
  thus, according to Lemma~\ref{lm:fjoin}(\ref{lm:fjoin_a}),
  \begin{equation*}\label{eq:notinST_17}
    v_p(f,[t_l,t_{l+1}]) = |f(t_{l})-f(t_{l+1})|^p.
  \end{equation*}   


  
  Symmetric argument could be used in other direction. 
  Firstly lets modify last inequality
  \begin{equation*}\label{eq:notinST_18}
    |f(y_{j-1})-f(t_k)+f(t_k)-f(t_{l+1})|^p > |f((y_{j-1})-f(t_{k})|^p 
      + v_p(f,[t_k,t_l]) +  v_p(f,[t_l,t_{l+1}]). 
  \end{equation*}   
  From the (\ref{eq:notinST_16}) we get that
   the following inequalities holds
  \begin{eqnarray*}
    f(t_k)-f(t_{l+1}) & \geq & 0, \\
    f(t_{k-1})-f(t_{k}) \geq f(y_{j-1})-f(t_k) & \geq & 0.
  \end{eqnarray*} 

  Thus, from Corollary~\ref{cor:convex} we get
  \begin{eqnarray*}
    |f(t_{k-1})-f(t_k)+f(t_k)-f(t_{l+1})|^p&>&
      |f(t_{k-1})-f(t_{k})|^p+v_p(f,[t_k,t_l])+v_p(f,[t_l,t_{l+1}]), \\
    |f(t_{k-1})-f(t_{l+1})|^p &>& v_p(f,[t_{k-1},t_{k}]) 
      + v_p(f,[t_k,t_l])+v_p(f,[t_l,t_{l+1}]).
  \end{eqnarray*}  
  As previous, the last inequality holds, 
  because $t_{k-1}$ and $t_{k}$ are
  $f$-joined.

  Finally, using Lemma~\ref{lm:pvarSubset} we conclude that
  $$|f(t_{k-1})-f(t_{l+1})|^p > v_p(f,[t_{k-1},t_{l+1}]).$$
  This contradicts with the definition of $p$-variation.    
  
  
\end{proof}

\subsection{Extra results}


\begin{definition}[Extremum]\label{def:extremum}
  We will call the point $t \in  [a,b]$ an \emph{extrema}
  of the function $f$ in interval $[a,b]$ if   
  $f(t)=\sup \left\{ f(z):z\in \left[ a,b\right]
  \right\} $ or 
  $f(t)=\inf \left\{ f(z):z\in \left[ a,b\right] \right\} $.
\end{definition}

%%% Global extremums are meaningful breaks %%% 
\begin{proposition}\label{prop:SplitMinMax}
  Let $f:[a,b] \rightarrow \mathbb{R}$ is $PM$. 
  If point $x \in [a,b]$ is extrema of the function $f$, 
  then $x \in \overline{SP}_p(f,[a, b])$.
\end{proposition}
\begin{proof}
  Let $f:[a,b] \rightarrow \mathbb{R}$ is $PM$ and
  point $x \in [a,b]$ is an extrema of the function $f$.
  Let $\{t_i\}_{i=0}^{n} \in SP_p(f,[a,b]$ 
  be any supreme partition. 
  Then
  \begin{equation}
    \exists j \in 1,\dots,n: x \in [t_{j-1},t_j] 
  \end{equation}  

  Since $x$ is an extrema of function $f$ in interval
  $[a, b] \supset [t_{j-1},t_j]$, 
  the point $x$ is an extrema in interval 
  $[t_{j-1}, t_j]$ as well. Therefore,
  \begin{equation}
    f(t_{j-1})-f(t_{j})|^p \leq |f(t_{j-1}) - f(x)|^p
  \end{equation}
  or
  \begin{equation}
    |f(t_{j-1})-f(t_{j})|^p \leq |f(x)-f(t_{j})|^p.
  \end{equation}
  As a result,
  \begin{eqnarray*}
  |f(t_{j-1})-f(t_{j})|^p 
    &\leq &|f(t_{j-1}) - f(x)|^p + |f(x) - f(t_{j})|^p,\\
  |f(t_{j-1}) - f(t_{j})|^p 
    &\leq & v_p(f, [t_{j-1}, x]) + v_p(f, [x, t_{j}]),\\
  v_p(f, [t_{j-1}, t_{j}])
    &\leq & v_p(f,[t_{j-1}), x]) + v_p(f,[x, t_{j}]).
  \end{eqnarray*}    
  The last inequality holds, because $t_{j-1}$ and  $t_{j}$
  are $f$-joined, so, 
  $v_p(f, [t_{j-1}, t_{j}])=|f(t_{j-1}) - f(t_{j})|^p$.
  Since $v_p(f, [t_{j-1}, t_{j}])
    < v_p(f,[t_{j-1}), x]) + v_p(f,[x, t_{j}])$
  is not valid, the equation
  \begin{equation}
    v_p(f, [t_{j-1}, t_{j}])
    = v_p(f,[t_{j-1}, x]) + v_p(f,[x, t_{j}])  
  \end{equation}
  holds.
  
  Moreover, applying Proposition~\ref{prop:pvar_sum_x} 
  for the partition
  $\{a, t_{j-1}, t_j, b \}$ we have 
   \begin{eqnarray*}
    v_p(f, [a, b]) &=& v_p(f,[a, t_{j}]) 
      + v_p(f,[t_{j-1}, t_{j}]) + v_p(f,[t_{j}, b]) \\
    &=& v_p(f,[a, t_{j}]) + v_p(f,[t_{j-1}, x]) 
      + v_p(f,[x, t_{j}]) + v_p(f,[t_{j}, b])
  \end{eqnarray*}
  From the same proposition follows, that
  $\exists \kappa \in SP_p(f, [a, b]):x\in\kappa$,
  so, by definition, $x \in \overline{SP}_p(f,[a, b])$.  
\end{proof}


\begin{lemma}\label{lm:evenkappa}
  Let $f:[a,b] \rightarrow \mathbb{R}$ be PM.
  If $K(f)$ is even, then
  \begin{equation}
    \exists x \in (a, b): x \in  \overline{SP}_p(f, [a, b]). 
  \end{equation}       
\end{lemma}
\begin{proof}
  Suppose $f:[a, b] \rightarrow \mathbb{R}$ is PM.
  Lets take any $\{x_i\}_{i=0}^n \in X(f)$.
  If $n>1$, then without loss of generality
  we can assume that Proposition~\ref{prop:PM_form}(a) 
  holds. So, $f(x_0)>f(x_1)<f(x_2)>\dots$, i.e.
  $f(x_{2i-1})<(x_{2i})$, thus,
  if $n$ is even, then $f(x_{n-1})<(x_{n})$,
  therefore, point $x_{n}$ is not a global minimum.
  Since $f(x_0)>f(x_1)$, point $x_0$ is also
  not a global minimum. As a result,
  global minimum is in interval $(a, b)$ 
  and it is in $\overline{SP}_p(f, [a, b])$ according to
  Lemma~\ref{prop:SplitMinMax}. 
\end{proof}


% Small increase
\begin{lemma}\label{lm:SI}
  Let $f:[a,b] \rightarrow \mathbb{R}$ be PM
  and $\{t_i\}_{i=0}^n \in PP[a, b]$.
  Suppose 
  \begin{equation}\label{eq:SI_1}
    \{t_i\}_{i=0}^{n-1} \in SP_p(f, [a, t_{n-1}]) 
  \end{equation} 
  and 
  \begin{equation}\label{eq:SI_2}
    \{t_i\}_{i=1}^{n} \in SP_p(f, [t_1, b]).
  \end{equation} 
  Then
  \begin{equation}\label{eq:SI_3}
    \forall i \in \{1,\dots,n-1\}: t_i \in \overline{SP}_p(f, [a, b])
  \end{equation} 
  or
  \begin{equation}\label{eq:SI_4}
    \forall i \in \{1,\dots,n-1\}: t_i \notin \overline{SP}_p(f, [a, b])
  \end{equation}   
\end{lemma}
\begin{proof}
  Let the assumptions of lemma be valid. 
  Firstly, suppose 
  \begin{equation}\label{eq:SI_p1}
    \exists j \in \{1,\dots,n-1\}: t_j \in \overline{SP}_p(f, [a, b]).
  \end{equation}
  This means that 
  $\exists \kappa \in SP_p(f, [a, b]): t_j \in \kappa$.
  Then, by Proposition~\ref{prop:pvar_sum_x},
  \begin{equation}
    v_p(f, [a, b]) = v_p(f, [a, t_j]) + v_p(f, [t_j, b])
  \end{equation}
  
  Since (\ref{eq:SI_1}) and (\ref{eq:SI_1}) holds, 
  by Lemma~\ref{lm:spSubset} we get 
  \begin{eqnarray}
    v_p(f, [a, t_j]) = s_p(f, \{t_i\}_{i=0}^j, \\
    v_p(f, [t_j, b]) = s_p(f, \{t_i\}_{i=j}^n.
  \end{eqnarray}
  
  Therefore,
  \begin{equation}
    v_p(f, [a, b]) = s_p(f, \{t_i\}_{i=0}^j) +  s_p(f, \{t_i\}_{i=j}^n)
      = s_p(f, \{t_i\}_{i=0}^n).
  \end{equation}  
  
  So, $\{t_i\}_{i=0}^n \in SP_p(f, [a, b])$, 
  thus, statement (\ref{eq:SI_3}) holds.
  
  On the other hand, if (\ref{eq:SI_p1}) is not valid, 
  then statement (\ref{eq:SI_4}) holds.  
  
\end{proof}



\section{p-variation calculus}  
  
This chapter will present the algorithm that calculates 
$p$-variation for the sample 
(see Def.~\ref{def:pvarseq}).
Nonetheless, this algorithm could be used to calculate
the $p$-variation for arbitrary piecewise monotone function. 
This algorithm will be called \emph{pvar}. 
It is already realised in the R (see~\cite{R}) package \emph{pvar} and
is publicly available on CRAN\footnotemark.
\footnotetext{
http://cran.r-project.org/web/packages/pvar/index.html
}

Suppose $X=\{X_{i}\}_{i=0}^{n}$ is any real-value sequence of numbers. 
We will call such sequence a \emph{sample}, 
whereas $n$ will be referred to as a \emph{sample size}.
The formal definition of $p$-variation 
of the sample is given in Definition~\ref{def:pvarseq}, 
which states that
\begin{equation}
  v_p(X) = v_p(G_X(t),[0,n]),
\end{equation}
where $G_X(t)$ is sample function defined in  
Definition~\ref{def:Seq2Fun}.

On the other hand, the algorithm does not use
$G_X(t)$ function directly, but it actually operates the
sample $X$. Therefore, in the context of algorithm,
it is more convenient to use the sample $X$ 
rather then the function $G_X(t)$. 
Thus, please don't be mislead then
in the context of sample we will 
refer to properties of $p$-variation 
that were proved to functions. 
In that case we actually have in mind function $G_X(t)$. 





It is worth noting, 
that the procedure \emph{pvar} could be used to calculate
the $p$-variation of any piecewise monotone function.
Let assume $f$ is any piecewise monotone function.  
According to definition (see~\ref{def:PM}) 
there are points $a=x_1<\dots<x_n=b$ for some finite $n$ such 
that $f$ is monotone on each interval $[x_{j-1},x_j],\;j=1,2,\dots,n$.
Lets construct the the sample $X$, 
which contains the values of the function $f$ 
at the points $\{x_{i}\}_{i=0}^{n}$, namely,
$X=\{X_{i}\}_{i=0}^{n}:=\{f(x_{i})\}_{i=0}^{n}$.
It is straight forward to see that 
$f \stackrel{PM}{=} G_X$, therefore, if $p \geq 1$ then $v_p(X)=v_p(G_X)=v_p(f)$.


\subsection{Main function}

The procedure \emph{pvar} that calculates
$p$-variation will be presented here. Firstly,
we will introduce the main schema, further, each step will
be discussed in more details. 

The main goal of the procedure is to find
any supreme partition 
$\kappa \in SP_p(G_X)$ and calculate $p$-variation
by $v_p(X) = s_p(G_X, \kappa)$.
The main principal of the procedure is to identify
points that are not in supreme partition and drop them out from
further consideration.

For the continence, 
it is worth pointing out, that 
$p$-variation of the sample could express as
\begin{equation}
  v_p(X) = \max\left\{ \sum_{i=1}^k |X_{j_i} - X_{j_{i-1}}|^p :
  0=j_0<\dots<j_k=n,\; k=1,\dots,n  \right\}.
\end{equation}
This expression could be verified from
Proposition~\ref{prop:sup_in_PM},
which states, that $p$-variation
is achieved in a subset of any partition $r \in X(f, [0,n])$.
So, we choose partition $r$ to be a subset
of $\{0, 1, \dots, n\}$. This partition exists, 
since all the values of function $G_X$ are
generate from points on $\{0, 1, \dots, n\}$, thus,
all the remaining points are redundant for $p$-variation calculus.
If $j \in \{0, 1, \dots, n\}$ then $G_X(j) = X_j$. Therefore,
instead of using $G_X(j)$ we can directly use
the sample member $X_j$. 

The sample members $X_0$, $X_n$ are in supreme partition according to
definition. All the other members $X_1, \dots, X_{n-1}$
may or may not be in supreme partition.
The number of all possible combinations are
$2^{n-1}$, but we will not investigate all of them.
Rather we will use the properties of $p$-variation
to identify the points that are not
in supreme partition and drop them out from further consideration.

Suppose $j \in \{0, 1, \dots, n\}$ and
$j \notin SP_p(G_X, [0, n])$. This means that
$j$ can not be in supreme partition. Therefore, 
sample member $X_j$ should be excluded from further consideration.
Removing $X_j$ means that we 
updating sample $X$ with new sample $X'$ which do not have
$X_j$ element, namely,  
$$
\{X'_{i}\}_{i=0}^{n-1}:=\{X_0,\dots,X_{j-1},X_{j+1},\dots,X_n\}.
$$

The members of the sample $X$ that are in supreme partition are
called \emph{significant}. All the members
that are not in supreme partition are called \emph{insignificant} and   could be removed from original sample without any effect to 
the value of $p$-variation.


The most important property which were used in algorithm is
stated in Proposition~\ref{prop:notinST}. This property
allows to investigate small intervals and find insignificant points.
Later on we will discus in great detail how we actually using is.
The main steps in \emph{pvar} procedure
goes as follows.


\paragraph{Procedure \emph{pvar}.}
Input: sample $X$, scalar $p$.

\begin{enumerate}
  \item \emph{Removing monotonic points}. According to 
  Proposition~\ref{prop:sup_in_PM}
  all points that are not the end of 
  monotonic interval could be excluded 
  from further consideration. 
  
  \item \emph{Checking all small intervals}. Every small intervals are
    checked if it is possible to 
    identify insignificant points. 
    This operation is based on 
    Proposition~\ref{prop:notinST} which states that 
    If $x \notin \overline{SP}_p(f, [a', b'])$, then 
    $x \notin \overline{SP}_p(f, [a, b])$.
    So, if $X_j$ appears to be insignificant in
    any small sub-sample, then
    we do not need to consider possibility for
    $X_j$ to be in supreme sample any more. 
    More details are given in section~\ref{sec:CheckSmallIntervals}.  
 
  \item \emph{Merging small intervals}.  
   Let $\kappa_{a,c} \in SP_p(G_X, [a,c])$ and
   $\kappa_{c,b} \in SP_p(G_X, [c, b])$. By using $\kappa_{a,c}$
   ans $\kappa_{c,b}$, we can effectively
   find $\kappa_{a,b} \in SP_p(G_X, [a, b])$, since
   $\kappa_{a,b} \subset \kappa_{a,c} \cup \kappa_{c,b}$. 
   Finding
   $\kappa_{a,b}$ from $\kappa_{a,c}$ and $\kappa_{c,b}$ 
   is called \emph{merging}.
   In this step we repeat merging of all small intervals until
   we get the finial supreme partition.
   The whole procedure is described in section~\ref{sec:meging}.  
  
\end{enumerate}

The corresponding psiaudocode of the main function goes as follows
\begin{lstlisting}
pvar <- function(x, p){
  partition <- Remove_Monotonic_Points(x) 
  partition <- Test_Points_In_Small_Intervals(partition, p, LSI = 3) 
  partition <- Merge_Small_Intervals(partition, p, LSI=3+1)
  pvar <- sum(abs(diff(partition))^p)  
  return(pvar)
}
\end{lstlisting}
  
\subsection{Detail explanation}

\subsubsection{Removing monotonic points}

According to Proposition~\ref{prop:sup_in_PM}
the $p$-variation is achieved 
on the partition of monotonic intervals.
Therefore the points that are not 
the end of monotonic interval are insignificant 
and could be drop out.
The points that are the end of monotonic intervals
will be referred as \emph{corners}. 
In one loop we can 
find all the corners by checking if the sequence
changed the direction form increasing to decreasing
(or vice versa).
The corresponding pseudo code looks as follows:

\begin{lstlisting}
INCLUDE x[0] to set of CORNERS 
SET direction to 0 
for(i = 1; i < n; ++i) {
  if(x[i-1]<x[i]){
    if(direction<0){
      INCLUDE x[i-1] to set of CORNERS 
    }     
    SET direction to 1 
  }
  if(x[i-1]>x[i]){
    if(direction>0){
      INCLUDE x[i-1] to set of CORNERS 
    }     
    SET direction to -1 
  }        
}  
INCLUDE x[n] to set of CORNERS 
\end{lstlisting}

This procedure is quite simple and
 can by performed very quickly, since it does not include
any cross checking and could be done in one loop.
In \emph{pvar} package all corners could be found with \emph{ChangePoints} function.



\subsubsection{Checking all small intervals}
\label{sec:CheckSmallIntervals}

Proposition~\ref{prop:pvar_sum_x} and 
Lemmas~\ref{lm:SI} and~\ref{lm:evenkappa} gives an effective 
way to identify part of insignificant points 
using quite simple operations. 
The pseudo code is given in the end the chapter and
now we will reviled it main principles.

Suppose $X$ is a sample without monotonic points. 
Let investigate a sub-sample in
any small interval of the length $m$ ($m=2,\dots,n-1$), 
namely, lets examine a sub-sample 
$\{X_{i}\}_{i=j}^{j+m}$
for any $j=0,...,n-m$.
According to Proposition~\ref{prop:pvar_sum_x},
If 
\begin{equation}\label{eq:dropsamll1}
  \sum_{i=j+1}^{j+m} |X_{i}-X_{i-1}|^p < |X_j-X_{j+m}|^p,
\end{equation}
then,,
partition $\{j,\dots,j+m\}$ 
can not be supreme partition in interval
$[j,j+m]$. 
Therefore, some of the points $\{X_{i}\}_{i=j+1}^{j+m-1}$
must be insignificant.
Moreover, Lemmas~\ref{lm:SI} and~\ref{lm:evenkappa} 
gives opportunity easily identify 
insignificant points. To do so, we must 
systemically investigate investigate small intervals starting form
small $m$.

\begin{itemize}

\item 
Let $m=2$. Then $|\{X_{i}\}_{i=j}^{j+2}|=2$, thus according to
Lemma~\ref{lm:evenkappa}, $\{X_{i}\}_{i=j}^{j+2}$ has at least one
inner significant point. 
Since it has only one inner point $X_{j+1}$, then 
this point must be significant. 
So, if $m=2$, then  (\ref{eq:dropsamll1}) is always invalid,
therefore, we don't need to do any checking.

\item 
Let $m=3$. Firstly, we need to check if (\ref{eq:dropsamll1}) holds.
If so,
then at least one 
point must be insignificant.
In addition, Lemma~\ref{lm:SI} ensures that in this case
both middle points $X_{t+1}$, $X_{t+2}$ 
are insignificant, therefore, they should be removed. 

We do this checking for  for all
$j=0,...,n-2$. In this way
we can identify quite an amount of insignificant
points. Moreover, after removing them from the sample the 
sample changes, so we can repeat the same procedure
and find more insignificant points. We do it until
no new insignificant points are 
identified.

\item 
If $m=4$, then $|\{X_{i}\}_{i=j}^{j+2}|$ is even, 
thus, according to Lemma~\ref{lm:evenkappa}, 
$\{X_{i}\}_{i=j}^{j+m}$ has at least one
inner significant point. So, all inner points are significant, 
based on Lemma~\ref{lm:SI}, therefore, 
(\ref{eq:dropsamll1}) do not hold.
Lemma~\ref{lm:evenkappa} ensure
that this argument holds in all cases, then $m$ is even.
So, actually we need to investigate only the cases, then 
$m$ is odd. 

\item 
Let 
$m=5$ and all sub-samples
 $\{X_{i}\}_{i=j}^{j+3},\;j=0,...,n-3$ are already 
 checked.
If (\ref{eq:dropsamll1}) holds, then, based on
Lemma~\ref{lm:SI}, all middle points
$X_{t+1}$, $X_{t+2}$, $X_{t+3}$  and $X_{t+4}$ are  
identified as insignificant and should be removed.
After removing points, the sample changes, thus,
if we want to apply this checking in again,
then we should start checking from $m=3$ again, because
we must make sure that all sub-samples
$\{X_{i}\}_{i=j}^{j+3},\;j=0,...,n-3$ are already 
checked, because this is a requirement of the Lemma \ref{lm:SI}.

\item If we want to go further and check
sub-samples for $m=7,9,11\dots$, every time we have to
start from $m=3$ and increase $m$ by $2$ only then
(\ref{eq:dropsamll1}) is not valid for
all sub-samples.

\end{itemize}

We can continue this checking until $m$ 
reaches an arbitrary threshold $M$.
The  pseudo-code  of this procedure is given below
\begin{lstlisting}

SET dum_X to NULL
WHILE dum_X is not equal to X AND m<M
  SET dum_X	to X
  SET m = 3
  CHECK all sub-samples with m=d 
  UPDATE X by dropping all insignificant points
  WHILE dum_X is equal to X AND	m<M
    SET m = m + 2
    CHECK all sub-samples with m=d points
    UPDATE X by dropping all insignificant points
  ENDWHILE  
ENDWHILE

\end{lstlisting}


With this procedure, we can actually find 
$p$-variation, setting $M$ to sufficiently large value.
But this way of finding $p$-variation is not effective, 
because in case of any removing 
we should go back and start checking
from $m=3$.
So, it is reasonable to use this procedure
only for small $m$ and then go
to final step of $p$-variation calculus.


% From Monte-Carlo experiment we made
% a recommendation that $m$ sho



\subsubsection{Merging small intervals}
\label{sec:meging}



Suppose it is already known that 
a sub-samples $\{X_{i}\}_{i=a}^{w}$  and $\{X_{i}\}_{i=w}^{b}$
are a supreme partitions in intervals $[a,w]$ and $[w,b]$ correspondingly. 
The \emph{merge} operation of intervals $[a,w]$ and $[w,b]$ is
an operation that finds supreme partition 
of the interval $[a,b]$ taking into account
that  $\{X_{i}\}_{i=a}^{w}$  and $\{X_{i}\}_{i=w}^{b}$
are a supreme partitions.

The final procedure in $p$-variation calculus will be
presented in this subsection. It is based on
merging operation.
All the small intervals which where already checked in the 
previous procedure are merged one by one until all the
possible combinations are checked and we end up having
the supreme partition and $p$-variation.

This procedure have two levels that will be discussed separately.

\paragraph{Merge two intervals}

This paragraph will present how two intervals are merged .
Suppose $\{X_{i}\}_{i=a}^{w}$  and $\{X_{i}\}_{i=w}^{b}$
are a supreme partitions in intervals $[a,w]$ and $[w,b]$ correspondingly.  
In order to find supreme partition in interval $[a,b]$
we need to investigate all possible
$f$-joints between points $X_t, t\in \{a,\dots,v-1\}$ and
$X_r, r\in \{v,\dots,b\}$ and choose pair of points
$(t, r)$ which maximise $s_p$ function, namely  
\begin{eqnarray}
  (t, r) &:=& \argmax_{(i,j) \in [a, v-1] \times [v, b] } 
    \left\{ \sum_{k=1}^i |X_{k-1} + X_{k}|^p + |X_i-X_j|^p 
    + \sum_{k=j+1}^b |X_{k-1} + X_{k}|^p \right\}, \nonumber \\
  &=& \argmax_{(i,j) \in [a, v-1] \times [v, b] }  
    \left\{|X_i-X_j|^p +  \sum_{k=a+1}^b |X_{k-1} + X_{k}|^p  
    - \sum_{l=i+1}^j |X_{l-1} + X_{l}|^p  \right\}, \nonumber \\
   \label{eq:tr_argmax}  
   &=& \argmax_{(i,j) \in [a, v-1] \times [v, b] }  
    \left\{|X_i-X_j|^p - \sum_{l=i+1}^j |X_{l-1} + X_{l}|^p  \right\},   
\end{eqnarray}


In addition, the calculation of $(t, r)$ in (\ref{eq:tr_argmax})
could be optimised, since we actually do not need to check
all $(i,j) \in [a, v-1] \times [v, b]$. 
Based on Lemma~\ref{lm:fjoin},
points $X_t$ and $X_r$ could be $f$-joined only if 
\begin{equation}
  \forall i \in \{t,\dots,r\}: X_t \geq X_i \geq X_r \text{ or } 
  X_t \leq X_i \leq X_r.
\end{equation} 
   
And the last statement holds only if 
\begin{equation}\label{eq:setT}
  X_t = \left\{ \min ( X_i : i \in \{t, \dots, w-1\}),
  \max ( X_i : i \in \{t, \dots, w-1\}) \right\}  
\end{equation}
and
\begin{equation}\label{eq:setR}
  X_r = \left\{ \min ( X_i : i \in \{w, \dots, r\}),
  \max (X_i : i \in \{w, \dots, r\}) \right\}.  
\end{equation} 

Let $T$ denotes a set of $i \in \{a,\dots ,w-1\}$ which satisfy 
(\ref{eq:setT}) and $R$ denotes a set of $j \in \{w,\dots ,b\}$ which satisfy (\ref{eq:setR}).
Then (\ref{eq:tr_argmax}) 
could be expressed as
\begin{equation}
  (t, r) :=  
    \argmax_{(i,j) \in T \times R }  
    \left\{|X_i-X_j|^p - \sum_{l=i+1}^j |X_{l-1} + X_{l}|^p  \right\}.   
\end{equation} 

Since points $X_t$ and $X_r$ are $f$-joint, all the 
pints in interval $t+1,\dots, r-1$ are insignificant and should be removed. And the remaining points are supreme partition.

\begin{lstlisting}

INITIALISE list T to w-1;
max_i = min_i = X_w
FOR i = w-2 TO a{
  if (X_i is greater then max_i){
    max_i = X_i
    include i in T  
  }
  if (X_i is less then min_i){
    max_i = X_i
    include i in T  
  }
}

INITIALISE list R to w;
max_i = min_i = X_w
FOR i = w-2 TO a{
  if (X_i is greater then max_i){
    max_i = X_i
    include i in R  
  }
  if (X_i is less then min_i){
    max_i = X_i
    include i in R  
  }
}

FOR (r in R){
  FOR (t in T){
    balance = |x_t - X_r|^p - sum(||^p
  
  }
}




\end{lstlisting}


\paragraph{Merging all small intervals}
dfd

\begin{lstlisting}

while the length of in > 1
  
  for  
    int[i] = merge(int[i], int[i+1])
    i+2;
  end for
  delete all even int
  
  

end 


\end{lstlisting}

\section{Conclusion}
  
%%%%%%%%%%%%%%%%%%%%%%%%%%%%%%%%%%%%%%%%%%%%%%%%%%%%%%%%%%%%%%%%%%%%%%%%%%%%%%%%%%%%%%%%%%% 
\begin{thebibliography}{99}  
  \bibitem{Qian} J. Qian. The $p$-variation of Partial Sum Processes
  and the Empirical Process // Ph.D. thesis, Tufts University, 1997.
  
   \bibitem{R} R
  
\end{thebibliography}






\end{document}
