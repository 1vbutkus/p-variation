
% Ar korektiska citoti jei yra perfrazuota?
% Ar korektiska citoti jei yra atskiras atvejis?


% Filosofinis klausimas - gal galima irodineti lenteles forma?
% lentele nebutina, tiesiog tokia forma, kur aiskiai matosi lygmenys. 

\documentclass[12pt, a4paper]{article}

\usepackage[utf8]{inputenc}

\usepackage{lmodern}
\usepackage{layouts}                    % layouto vienetu transformacijos  
\usepackage[hyperref]{ntheorem}         % leidzia pakesti theoremu stiliu

\usepackage{amssymb}
\usepackage{amsmath}
\usepackage{listings}            % graziam kodui
\usepackage{enumerate}



% \usepackage{caption}                    % paveikleliu dejimas i veina vieta
% \usepackage{subcaption}                 % paveikleliu dejimas i veina vieta


% \usepackage{microtype}				          % isjungia triukus, kuriu nereikia
% \DisableLigatures[f]{encoding = *, family = * }

\usepackage{ifpdf}
\ifpdf
\usepackage{pdfpages}
\usepackage[pdftex]{hyperref}
\fi

\usepackage{color}
\definecolor{mygreen}{rgb}{0,0.6,0}
\definecolor{mygray}{rgb}{0.5,0.5,0.5}
\definecolor{FontGray}{rgb}{0.96,0.96,0.96}
\definecolor{mymauve}{rgb}{0.58,0,0.82}

\definecolor{tbred}{rgb}{1,0.6,0.6}
\definecolor{tbgreen}{rgb}{0.6,1,0.6}


\lstset{ %
  backgroundcolor=\color{FontGray},   % choose the background color; you must add \usepackage{color} or \usepackage{xcolor}
  basicstyle=\footnotesize,        % the size of the fonts that are used for the code
  breakatwhitespace=false,         % sets if automatic breaks should only happen at whitespace
  breaklines=true,                 % sets automatic line breaking
  captionpos=b,                    % sets the caption-position to bottom
  commentstyle=\color{mygreen},    % comment style
  deletekeywords={...},            % if you want to delete keywords from the given language
  escapeinside={\%*}{*)},          % if you want to add LaTeX within your code
  extendedchars=true,              % lets you use non-ASCII characters; for 8-bits encodings only, does not work with UTF-8
  frame=single,                    % adds a frame around the code
  keepspaces=true,                 % keeps spaces in text, useful for keeping indentation of code (possibly needs columns=flexible)
  keywordstyle=\color{blue},       % keyword style
  language=R,                      % the language of the code
  morekeywords={*,...},            % if you want to add more keywords to the set
  numbers=left,                    % where to put the line-numbers; possible values are (none, left, right)
  numbersep=5pt,                   % how far the line-numbers are from the code
  numberstyle=\tiny\color{mygray}, % the style that is used for the line-numbers
  rulecolor=\color{black},         % if not set, the frame-color may be changed on line-breaks within not-black text (e.g. comments (green here))
  showspaces=false,                % show spaces everywhere adding particular underscores; it overrides 'showstringspaces'
  showstringspaces=false,          % underline spaces within strings only
  showtabs=false,                  % show tabs within strings adding particular underscores
  stepnumber=1,                    % the step between two line-numbers. If it's 1, each line will be numbered
  stringstyle=\color{mymauve},     % string literal style
  tabsize=2,                       % sets default tabsize to 2 spaces
  title=\lstname                   % show the filename of files included with \lstinputlisting; also try caption instead of title
}
 
 
 
% stanratiniu fontu nustatymas
% \renewcommand{\sfdefault}{uhv}
% \renewcommand{\rmdefault}{utm}
% \renewcommand{\ttdefault}{ucr}


\theorembodyfont{\normalfont}
\theoremseparator{.}
\newtheorem{theorem}{Theorem}[section]
\newtheorem{lemma}[theorem]{Lemma}
\newtheorem{proposition}[theorem]{Proposition}
\newtheorem{definition}[theorem]{Definition}
\newtheorem{corollary}[theorem]{Corollary}


\newenvironment{proof}[1][Proof]{\noindent \textbf{#1.} }{\  \rule{0.5em}{0.5em}}
\numberwithin{equation}{section}

\title{The calculus of $p$-variation }



\begin{document}
  \maketitle


\begin{itemize}  
  \item $PP[a,b]$ --  the set of all point partitions of $[a,b]$ (def.~\ref{def:pp}).
  \item $s_{p}(f;\kappa)$ --  $p$-variation sum (def.~\ref{def:pvar}).
  \item $v_{p}\left( f\right)$ -- $p$-variation of the function $f$ (def.~\ref{def:pvar}).
  \item $MP_{p}(f,[a,b])$ -- the set of meaningful partitions (def.~\ref{def:pvar}).
  \item $PM[a,b]$ -- a set of piecewise monotone functions (def.~\ref{def:PM}).
  \item $CPM[a,b]$ -- a set of continuous piecewise monotone functions (def.~\ref{def:PM}).
  \item $K(f,[a,b])$ -- minimal size of PM partitions (prop.~\ref{prop:K_f}).
  \item $X(f,[a,b])$ -- the set of PM partitions with minimal size (def.~\ref{def:X_f}).
  \item $\overline{BP}(f,[a,b])$ -- the set of break points (definition \ref{def:BreakPoint}).
  \item $\overline{BP}(f,(a,b))$ -- the set of inner break points (def.~\ref{def:BreakPoint}).
  \item $BP(f,[a,b])$ -- the set of partitions that contains only break points (def.~\ref{def:BreakPoint}). 
  \item $BP(f,(a,b))$ -- the set of partitions that contains only break points, but excludes the partition $\{a,b\}$. (def.~\ref{def:BreakPoint}). 
  \item $\overline{MB}_{p}(f,[a,b])$ -- the set of meaningfully break points (def.~\ref{def:MB} ).
  \item $MB_{p}(f,[a,b])$ -- the set of partitions that contains only meaningfully break points (def.~\ref{def:MB} ).
\end{itemize}   


\section{Mathematical analysis}

\subsection{General known properties}
  
  
%%% Definition of partition %%% 
% Status
\begin{definition}[Partition]\label{def:pp}
  Let $J = [a,b]$ be a closed interval of real numbers with $-\infty < a \leq b <+\infty$. 
  If $a < b$, an ordered set $\kappa = \{x_{i}\}_{i=0}^{n}$ of points in $[a,b]$ such that a  
  $a=x_{0}<x_{1}<x_{2}<...<x_{n}=b$ is called a \emph{(point) partition}. 
  The size of the partition is denoted $\left|\kappa \right| :=\#\kappa-1=n$. 
  The set of all point partitions of $[a,b]$ is denoted by $PP[a,b]$.  
\end{definition} 
 
%%% Definition of $p$-variation %%% 
% Status:OK
\begin{definition}[$p$-variation]\label{def:pvar}
  Let $f:[a,b] \rightarrow \mathbb{R}$ be a real function from an interval $[a,b]$.
  If $a<b$, for  $\kappa =\{x_{i}\}_{i=0}^{n} \in PP[a,b]$ the \emph{$p$-variation sum} is 
  \begin{equation}\label{eq:def_pvarsum}
    s_{p}(f,\kappa):=\sum_{i=1}^{n}| f( x_{i}) -f( x_{i-1})|^{p},
  \end{equation}
  where $0<p<\infty $. Thus, the \emph{$p$-variation} of $f$ over $[a,b]$ is $0$ if $a = b$ and otherwise
  \begin{equation}\label{eq:def_pvar}
    v_{p}(f) =v_{p}(f,[a,b]) :=\sup \left\{
    s_{p}(f,\kappa ):\kappa \in PP[a,b]\right\}.    
  \end{equation}
  
  The partition $\kappa$ is called \emph{meaningful} (in the content of $p$-variation) 
  if it satisfies the property $v_{p}(f) = s_{p}(f,\kappa)$.
  The set of such partitions is denoted
  $MP_{p}(f,[a,b])$.
\end{definition}  
 
%%% THe properties of $p$-variation %%% 
% Status:OK 
\begin{lemma}[Elementary properties]\label{lm:element_properties} 
  Let $f:[a,b] \rightarrow \mathbb{R}$ and $0<p<\infty $. 
  Then the following $p$-variation properties holds
  \begin{enumerate}[a)]
    \item \label{lm:ep_a} $ v_p(f,[a,b]) \geq 0 $,
    \item \label{lm:ep_b} $ v_p(f,[a,b]) = 0 \Leftrightarrow f \equiv Const. $,
    \item \label{lm:ep_c} $ \forall C \in \mathbb{R}: v_p(f+C,[a,b]) = v_p(f,[a,b]) $,  
    \item \label{lm:ep_d} $ \forall C \in \mathbb{R}: v_p(Cf,[a,b]) = C^p v_p(f,[a,b]) $,  
    \item \label{lm:ep_e} $ \forall c \in [a,b]: v_p(f,[a,b]) \geq v_p(f,[a,c]) + v_p(f,[c,b]) $,  
    \item \label{lm:ep_f} $ \forall [a',b'] \subset [a,b] : v_p(f,[a,b]) \geq v_p(f,[a',b']) $.  
    \item \label{lm:ep_g} $ \forall \kappa \in PP[a,b]:s_{p}(f;\kappa) \leq v_p(f,[a,b]) $.  
  \end{enumerate}

  All listed properties are elementary derived directly form the $p$-variation definition.
\end{lemma}
 
 

%%% Regulated function (definition) %%% 
% Status:OK  
\begin{definition}[Regulated function](\cite{Qian}, Def. 3.1)
  For any interval $J$, which may be open or closed at either end, bound or unbound, 
  real function $f$ is called \emph{regulated} on $J$ if it has left and right limits   
  $f(x-)$ and $f(x+)$ respectively at each point $x$ in interior of $J$,
  a right limit at the left end point and a left limit at the right endpoint.
\end{definition}  
 
%%% Limit points %%% 
% Status:OK   
\begin{proposition}(\cite{Qian}, Lemma 3.1)\label{prop:LimPoints} 
  Let $1 \leq p < \infty$. If $f$ is regulated then $v_{p}(f)$ remains the same
  if points $x+$, $x-$ are allowed as partition points $x_i$ in the definition \ref{def:pvar}.
\end{proposition}  
 
%%% Piecewise monotone functions (definition) %%% 
% Status:OK    
\begin{definition}[Piecewise monotone functions](\cite{Qian}, Def. 3.2)\label{def:PM}
  A regulated real-valued function $f$ on closed interval $[a,b]$ 
  will be called \emph{piecewise monotone} (PM)
  if there are points $a=x_0<\dots<x_k=b$ for some finite $k$ such 
  that $f$ is monotone on each interval $[x_{j-1},x_j],\;j=1,\dots,k$. 
  Here for $j=1,\dots,k-1$, $x_j$ may be point $x-$ or $x+$.
  The set of all piecewise monotone functions is denoted $PM=PM[a,b]$.
  
  In addition to PM, if $f$ is continuous function we will call it 
  continuous piecewise monotone(CPM). 
  The set of such functions is denoted $CPM=CPM[a,b]$. 
\end{definition}   


 
%%% Minimal size of the partitions %%% 
% Status:OK    
\begin{proposition}(\cite{Qian}, Prop. 3.1)\label{prop:K_f}
  If $f$ is PM, there is a minimal size of partition $|\kappa|$ for which the definition \ref{def:PM} holds.
  The minimal size of the the PM partition is denoted $K(f,[a,b])=K(f)$, namely
  \begin{equation}
  K(f):=\min\left\{n:\exists \{x_{i}\}_{i=0}^{n} \in PP[a,b]:  f \text{ is monotonic in each } [x_{j-1}, x_j]   \right\}.
  \end{equation}
\end{proposition}    
 
%%% The set of minimal partitions %%% 
% Status:OK  
\begin{definition}[The set of PM partitions with minimal size](\cite{Qian}, Def. 3.3)\label{def:X_f}
  If $f$ is PM, let $X(f)=X(f,[a,b])$ be the set of all $\{x_{i}\}_{i=0}^{K(f)}$ for which the definition 
  of PM (def.~\ref{def:PM}) holds. $X(f)$ is called the \emph{set of PM partitions with minimal size}.
\end{definition}    
 
%%% uniqueness of difference of PM partition %%%
% Status:OK  
\begin{proposition}(\cite{Qian}, Prop. 3.3)\label{prop:PM_Unique}
  Let $f$ is PM then the numbers 
  $\alpha_j(f):=f(x_j) - f(x_{j-1})$ for
  $\{ x_j \}_{j=0}^{K(f)} \in X(f)$ and 
  $j=1,2,\dots K(f)$
  are uniquely determined.
\end{proposition}  
 
 
%%% Equality by PM %%% 
% Status:OK 
\begin{definition}[The equality by PM](\cite{Qian}, Def. 3.4)\label{def:EqPM}
  If $f$, $g$ are two PM functions, possibly on different intervals, such that 
  $K(f)=K(g)$ and $\alpha_j(f)=\alpha_j(g)$ for $j=1,2,\dots,K(f)$, then we say that
  $f \stackrel{PM}{=} g$.
\end{definition}    
  
%%% f {DM}= g %%%
% Status:OK 
\begin{proposition}(\cite{Qian}, Cor. 3.1)\label{prop:f_PM_g}
  Let $p>1$ and functions $f$ and $g$ are PM. 
  If $f \stackrel{PM}{=} g$ or $f \stackrel{PM}{=} -g$, then $v_p(f)=v_p(g)$.
\end{proposition} 

%%% Exists meaningful partition in set of PM partition. %%%
% Status:OK 
\begin{proposition}(\cite{Qian}, Them. 3.1)\label{prop:sup_in_PM}
  Let $f$ is PM, $\kappa \in X(f)$ and $1 \leq p < \infty$. 
  Then the supremum of $p$-variation in Definition~\ref{def:pvar}
  is attained for some partition $r \subset  \kappa$.
\end{proposition} 
\begin{corollary}
  The set $MP_{p}(f,[a,b])$ is not empty for all $f \in PM[a,b]$.  
\end{corollary}

%%% Partial sum (definition) %%%
% Status:OK 
\begin{definition}[Partial sum]
  Let $X_1, X_2,\dots,X_n$ be any real-value sequence. 
  The \emph{partial sum} of the first $j$ terms is defined by  
  \begin{equation}
    S_j:=\sum_{i=1}^j X_i, \; j=1,2,\dots,n.
  \end{equation}
  In addition, lets denote $S_0=0$.  
\end{definition}

%%% Sample function (definition) %%%
% Status:OK 
\begin{definition}[Sample function]\label{def:Seq2Fun}
  Suppose $X=\{X_{i}\}_{i=0}^{n}$ is any real-value sequence. 
  We will call such sequence a \emph{sample}, 
  whereas $n$ will be referred to as a \emph{sample size}.
  Then the \emph{sample function}  
  $G_X:[0,n] \rightarrow \mathbb{R}$ is defined as 
  \begin{eqnarray}
    G_X(t) := X_{\lfloor t \rfloor},\;t\in[0,n].
  \end{eqnarray}
\end{definition} 

%%% $p$-variation of the sequence (definition) %%%
% Status:OK 
\begin{definition}[$p$-variation of the sequence]\label{def:pvarseq}
  Let $X=\{X_i\}_{i=0}^n$. The $p$-variation
  of the sample $X$ is defined as $p$-variation of the function $G_X(t)$, namely
  \begin{equation}
    v_p(X) := v_p(G_X(t),[0,n]).
  \end{equation}
\end{definition}

%%% $p$-variation of the sequence (definition) %%%
% Status:??? 
\begin{definition}[Piecewise linear function]\label{def:poligon}
  Let $X=\{X_i\}_{i=0}^n$ be any real-value sequence. 
  The function $L_X:[0,n] \rightarrow \mathbb{R}$ is defined as
  \begin{eqnarray}
  L_X(t) := (1+\lfloor t \rfloor-t)X_{\lfloor t \rfloor} + (t-\lfloor t \rfloor)X_{\lfloor t \rfloor+1}   ,\;t\in[0,n]
  \end{eqnarray}
  is called \emph{piecewise linear function}.
\end{definition}


%%%%%%%%%%%%%%%%%%%%%%%%%%%%%%%%%%%%%%%%%%%%%%%%%%%%%%%%%%%%%%%%%%%%%%%
\subsection{General properties with proofs}


%%% For all PM exists CPM that v(PM)=v(CPM) %%%
% Status: OK
\begin{proposition}\label{prop:PM2CPM}
  For all $f \in PM$ exists $g \in CPM$ such that $f \stackrel{PM}{=} g$.
\end{proposition} 
\begin{proof}
  Let $f$ be PM. According to Proposition~\ref{prop:PM_Unique} the
  numbers $\alpha_j(f):=f(x_j) - f(x_{j-1})$, $j=1,2,\dots K(f)$
  are uniquely determined.
  Then, the partial sums of the sequence $\alpha_j(f)$ are
  \begin{equation}
    S_j:=\sum_{i=1}^j \alpha_i(f).
  \end{equation}    

  Lets connect points $S_j$ by 
  piecewise linear function~(def.~\ref{def:poligon}), namely
  \begin{equation}
  L_S(t) := (1+\lfloor t \rfloor-t)S_{\lfloor t \rfloor} + (t-\lfloor t \rfloor)S_{\lfloor t \rfloor+1}   ,\;t\in[0,K(f)]
  \end{equation}  
   
  Function $L_S$ is CPM. In addition, 
  It is straight forward to see that 
  \begin{equation}
  \alpha_j(L_S)=S_j-S_{j-1}=\alpha_j(f),
  \end{equation}
  hence, by Definition~\ref{def:EqPM} $f \stackrel{PM}{=} L_S$.  
\end{proof}
\begin{corollary}
  Similar argument could be used to show that
  \begin{equation}
    G_X(t)\stackrel{PM}{=}L_X(t).
  \end{equation}
  Therefore, according to Proposition~\ref{prop:f_PM_g}, if $p>1$ then
  \begin{equation}
    v_p(X) := v_p(G_X(t),[0,n])= v_p(L_X(t),[0,n])
  \end{equation}
\end{corollary}


%%% VIP: p-variation is additive in meaningful points %%%
% Status: 
\begin{proposition}\label{prop:pvar_sum_x} 
  Suppose $f:[a,b] \rightarrow \mathbb{R}$ is CPM.
  Let $\{x_i\}_{i=0}^n \in PP[a,b]$ be any partition of interval $[a,b]$. 
  Then the proposition
  \begin{equation}
    \exists \kappa \in MP_{p}(f,[a,b]):\forall i,x_i \in \kappa
  \end{equation}
  is equivalent to 
  \begin{equation}
    v_p(f,[a,b]) = \sum_{i=1}^n v_p(f,[x_{i-1},x_i]).
  \end{equation}
\end{proposition}
\begin{proof}
  Necessary. Let $f:[a,b] \rightarrow \mathbb{R}$ be $CPM$,
  $\{x_i\}_{i=0}^n \in  PP[a,b]$ and 
  \begin{equation}
    \exists \kappa \in MP_{p}(f,[a,b]):\forall i,x_i \in \kappa. 
  \end{equation}
  
  Points from the partition $\kappa$ will be denoted $t_i$, i.e. $\kappa=\{t_i\}_{i=0}^m$.
  Then, according to definitions of $MP_{p}$ and $p$-variation (def.~\ref{def:pvar}) the following equation holds
  \begin{equation}\label{eq:ir1.eq1}
  v_p(f,[a,b])=s_{p}(f;\kappa)=\sum_{i=1}^{|\kappa|}|f(t_{i-1})-f(t_i)|^p 
    = \sum_{i=1}^n\sum_{j=h(i-1)+1}^{h(i)}|f(t_{j-1})-f(t_j)|^p,
  \end{equation}  
  where $h:\{0,1,\dots,n\} \rightarrow \{0,1,\dots,m\}$ denotes a function 
  from the set of index of $x$ to the set of index of $t$, namely:
  \begin{equation}\label{eq:ir1.eqx1}
    h(i):=(j_i : x_i = t_{j_i}=t_{h(i)}).
  \end{equation}  
  The equation (\ref{eq:ir1.eq1}) holds, because it's just a rearranged sum.
  
  Moreover, the inequality  
  \begin{equation}\label{eq:ir1.eq2}
  \sum_{j=h(i-1)+1}^{h(i)}|f(t_{j-1})-f(t_j)|^p \leq v_p(f,[x_{i-1},x_i])
  \end{equation}  
  holds according to Lemma \ref{lm:element_properties}(\ref{lm:ep_g}).
  
  As a result of (\ref{eq:ir1.eq1}) and (\ref{eq:ir1.eq2}) we get
  \begin{equation}\label{eq:ir1.eq3}
    v_p(f,[a,b]) \leq \sum_{i=1}^n v_p(f,[x_{i-1},x_i]).
  \end{equation}
  
  On the other hand, according to the same Lemma~\ref{lm:element_properties}(\ref{lm:ep_g}) the following inequality holds
  \begin{equation}\label{eq:ir1.eq4}
    v_p(f,[a,b]) \geq \sum_{i=1}^n v_p(f,[x_{i-1},x_i]).
  \end{equation}
  
  Finally, from the (\ref{eq:ir1.eq3}) and (\ref{eq:ir1.eq4}) follows 
  \begin{equation}\label{eq:ir1.eq5}
    v_p(f,[a,b]) = \sum_{i=1}^n v_p(f,[x_{i-1},x_i]).
  \end{equation}
  
  Sufficiency. Suppose $f:[a,b] \rightarrow \mathbb{R}$ is $CPM$ and 
  \begin{equation}\label{eq:ir1.eq6}
    v_p(f,[a,b]) = \sum_{i=1}^n v_p(f,[x_{i-1},x_i]).
  \end{equation}
  
  Lets take any partition from each of the sets $MP_{p}(f,[x_{i-1},x_i]),\;i=1,\dots,n$ and denote it $\kappa_i$.
  Then, let define a joint partition $\kappa:=\cup_{i=1}^n \kappa_i$. 
  Points from the partition $\kappa$ will be denoted by $t_i$. 
  In addition, we will use the function $h$, which is defined in (\ref{eq:ir1.eqx1}).
  Then, continuing the equation (\ref{eq:ir1.eq6}) we get
  \begin{equation}\label{eq:ir1.eq7}
  v_p(f,[a,b]) = \sum_{i=1}^n\sum_{j=h(i-1)+1}^{h(i)}|f(t_{j-1})-f(t_j)|^p
    =\sum_{i=1}^{|\kappa|}|f(t_{i-1})-f(t_i)|^p.
  \end{equation}
  This means that $\kappa \in MP_{p}(f,[a,b])$. Moreover, $\forall i:x_i \in \kappa$,
  because $\kappa=\cup_{i=1}^n \kappa_i$.  
\end{proof}

\subsection{Meaningful break points}

%%% Lexicographical order (definition %%%
% Status: OK
\begin{definition}[Lexicographical order]\label{def:LexiOrder}
  Suppose $f$ is CPM. Let $\kappa$ and $r$ denotes two partitions with minimal size, namely
  $\kappa=\{x_j\}_{j=0}^{K(f)}\in X(f)$ 
  and $r=\{ y_j \}_{j=0}^{K(f)}\in X(f)$.
  Then partitions $\kappa$ and $r$ could be compare using \emph{lexicographical order}, namely
  we will say that $\kappa \succ r$, if 
  \begin{itemize}
    \item $\exists j: x_j > y_j$; and
    \item $\forall i<j: x_i = y_i$.
  \end{itemize} 
  We will say $\kappa = r$ if $\forall j:x_j=y_j$.\\
  We will say $\kappa \succeq r$ if $\kappa \succ r$ or $\kappa = r$.
\end{definition}


%%% Lexicographical order is total %%%
% Status: Need reference
\begin{proposition}
  Binary relation $\succeq$ is a total order, i.e. the relationship is transitive, antisymmetric and total.
\end{proposition}
\begin{proof}  
   Need reference ...
\end{proof}  

%%% Exists maximal element %%%
% Status: OK
\begin{proposition} \label{prop:kappaf}
  Let $f:[a,b] \rightarrow \mathbb{R}$ is CPM. Then there is a partition  
  $\kappa_f$ such that
  \begin{equation}\label{eq:Kappaf}
    \forall r \in X(f): \kappa_f \succeq r .
  \end{equation}
\end{proposition}
\begin{proof}  
  Firstly, we will construct $\kappa_f$, latter, we will proof
  that the property (\ref{eq:Kappaf}) holds.
  
  Suppose,  $f:[a,b] \rightarrow \mathbb{R}$ is CPM. Let denote
  \begin{equation*}
    x_1:=\sup \left\{x\in[a,b]:|f(y_1)-f(a)| \leq |f(y_2)-f(a)|, \forall a \leq y_1 < y_2 \leq x \right\}.
  \end{equation*} 
  
  Other $x_i,\;i=2,\dots,K(f)$ will be defined by induction. 
  Suppose $x_{i-1}$ is defined and $x_{i-1}<b$, then 
  \begin{equation*}
    x_i:=\sup \left\{x\in[x_{i-1},b]:|f(y_1)-f(x_{i-1})| \leq |f(y_2)-f(x_{i-1})|, \forall x_{i-1} \leq y_1 < y_2 \leq x \right\}.
  \end{equation*}   
  
  According condition used in the definition of $x_i$,
  all $(x_{i-1},x_i),\;i=1,\dots,K(f)$ are monotonic intervals.

  In addition, points $x_i$ have the greatest values, 
  that satisfies the condition. 
  Therefore, if the values $x_i$ would increased,
  the condition would no longer be valid, i.e.   
  \begin{equation}\label{eq:KapaS.1}
   \forall i, \exists \delta: |f(x_i)-f(x_{i-1})|> |f(x_i+\varepsilon)-f(x_{i-1})|\text{, if } \varepsilon \in (0,\delta). 
  \end{equation}   
  
  We will show that  $\kappa_f:=\{x_i\}_{i=1}^{n}$
  satisfies the condition  (\ref{eq:Kappaf}). 
  Suppose to the contrary that
  \begin{equation}
   \exists r=\{t_i\}_{i=1}^{n} \in X(f): r \succ \kappa_s.
  \end{equation}    
  According to Definition of $\succ$ we get
  \begin{equation}
   \exists j: t_j>x_j\text{, and } t_i=x_i \text{, if } i<j.
  \end{equation}     
  
  Since $r \in X(f)$, the intervals $[t_{j-1}, t_j]$ 
  are monotonic intervals, i.e.
  \begin{equation}\label{eq:KapaS.2}
    |f(y_1)-f(t_{j-1})| \leq |f(y_2)-f(t_{j-1})|,
  \end{equation}      
  if $t_{j-1} \leq y_1 < y_2 \leq t_j$.
  But (\ref{eq:KapaS.2}) contradicts (\ref{eq:KapaS.1}), then
  $y_1=x_j$ and $y_2=x_j+\varepsilon$.  
\end{proof}     
  
%% f:[BP_1,BP_2] -> R is not a constant.
% Status: OK
\begin{lemma}\label{lem:noncons}
  Suppose $f:[a,b] \rightarrow \mathbb{R}$ is CPM and $\kappa_f=\{x_i\}_{i=1}^{K(f)}$ 
  is the partition from (\ref{eq:Kappaf}).
  Then, for all $x_i,i=1,\dots,K(f)-1$, the following proposition holds
  \begin{equation}
    \forall \varepsilon>0: f \text{ is not a constant in } [x_i,x_i + \varepsilon].
  \end{equation}
\end{lemma}
\begin{proof} 
  Suppose $f:[a,b] \rightarrow \mathbb{R}$ is CPM and $\kappa_f=\{x_i\}_{i=1}^{K(f)}$ 
  is the partition from (\ref{eq:Kappaf}).
  Suppose to the contrary, that exists $x_j$ and $\varepsilon > 0$ such that
  $f \text{ is not a constant in } [x_j,x_j + \varepsilon]$.
  Since $f$ is constant in $[x_j,x_j + \varepsilon]$,
  the point $x_j + \varepsilon$ is also the end of monotone interval (def.~\ref{def:PM}).
  
  Lets define the partition $\{y_i\}_{i=1}^{K(f)} \in X(f)$, which is
  \begin{equation}
    y_i :=
    \begin{cases}
      x_i & \text{, if } j \neq i \\
      x_i+\varepsilon  & \text{, if } j = i
    \end{cases},
  \end{equation}
  where $\varepsilon \in (0, x_{j+1}-x_j)$.
  
  It is straight forward to see that $\{y_i\}_{i=1}^{K(f)} \succ \kappa_f$, since
  $y_j>x_j$ and $y_i=x_i,i<j$. 
  This contradict the definition of $\kappa_f$. 
\end{proof}   
  
  
%%% Break points (definition) %%%
% Status:OK  
\begin{definition}[Break points]\label{def:BreakPoint}
  Let $\kappa_f$ be a partition form Proposition~\ref{prop:kappaf}.  
  The points $x_j \in \kappa_f$ will be called \emph{break points} 
  and the set of such points are denoted $\overline{BP}(f,[a,b])$, which is non empty and finite.
  The set of partitions that contains only break points are denoted $BP(f,[a,b])$, i.e.
  \begin{equation}
    BP(f) = BP(f,[a,b]) := \left\{\kappa \in PP[a,b]: \kappa \subset \overline{BP}(f,[a,b]) \right\}. 
  \end{equation}
  $BP(f,[a,b])$ is also non empty and finite.
  
  For the convenience let define the subsets of the sets $\overline{BP}$ and $BP$ 
  that excludes the ends of the interval, namely
  \begin{eqnarray}
    \overline{BP}(f,(a,b)) &:=& \overline{BP}(f,[a,b])\setminus \{a,b\}, \\
    BP(f,(a,b)) &:=& BP(f,[a,b])\setminus \{\{a,b\}\}.  \label{eq:BP(f,(a,b))}
  \end{eqnarray}
\end{definition}  
  
 
  
%%% meaningul partition ar in break points %%%
% Status: OK
\begin{proposition}\label{prop:pvar-max}
  Let $f$ is CPM function defined in $[a,b]$ and $1 \leq p < \infty$.
  Then the $p$-variations could be expressed as 
  \begin{equation}
    v_{p}\left( f\right) =v_{p}\left( f,[a,b]\right) =\max \left\{
    s_{p}(f;\kappa ):\kappa \in BP(f,[a,b]) \right\}.
  \label{pv1}
  \end{equation}
\end{proposition}
\begin{proof}
  This result follows directly from the Proposition~\ref{prop:sup_in_PM}.
\end{proof} 




%%% Meaningfull break points (definition) %%%
% Status: OK
\begin{definition}[Meaningfull break points]\label{def:MB} 
  Suppose $f$ is CPM and $1 \leq p < \infty$. 
  Let denote the set of \emph{partitions of meaningful break} (points)
  \begin{equation}
    MB_{p}(f) = MB_{p}(f,[a,b]):= MP_p(f,[a,b]) \cap BP(f, [a,b]).
  \end{equation}  
  The set $MB_{p}(f,[a,b])$ is not empty according to Proposition~\ref{prop:pvar-max}.
  
  The point $x$ will be called \emph{meaningful break} if $\exists \kappa \in MB_{p}(f): x \in \kappa$.
  The set of such points are denoted $\overline{MB}_p(f) = \overline{MB}_p(f,[a,b])$.
  The point $a$ is called \emph{meaningless} if 
  $a \notin \overline{MB}_{p}(f,[a,b]) $. 
\end{definition}

%%% The interval on meaningful breaks are additive in all possible combinations %%%
% Status: OK
\begin{lemma}\label{prop:pvarSubset}
  Let $\{x_i\}_{i=0}^{n} \subset \kappa \in MB_{p}(f,[a,b])$, then
  \begin{equation}
    \forall i,j: v_p(f,[x_i,x_j])= \sum_{k=i+1}^j v_p(f,[x_{k-1},x_k]),\; 0 \leq i<j\leq n. 
  \end{equation}
\end{lemma}
\begin{proof}
  Suppose $\{x_i\}_{i=0}^{n} \subset \kappa \in MB_{p}(f,[a,b])$. 
  Let choose $i$ and $j$ such that $0\leq i<j\leq n$.
  Because $MB_{p} \subset MP_{p}$, we can apply Proposition \ref{prop:pvar_sum_x} for the partition
  $\{x_0,x_i,x_{i+1},\dots,x_{j-1},x_j,x_n\}$. Thus,
  \begin{equation}
    v_p(f,[a,b]) = v_p(f,[x_{0},x_i]) + \sum_{k=i+1}^j v_p(f,[x_{k-1},x_k]) + v_p(f,[x_j,x_n]). 
  \end{equation}
  In addition, we can apply the same proposition for the partition $\{x_0,x_i,x_j,x_n\}$, then
  \begin{equation}
    v_p(f,[a,b]) =  v_p(f,[x_{0},x_i]) + v_p(f,[x_i,x_j]) + v_p(f,[x_j,x_n]). 
  \end{equation}
  By subtracting one equation form the other we get the result that
  \begin{equation}
    v_p(f,[x_i,x_j])= \sum_{k=i+1}^j v_p(f,[x_{k-1},x_k]). 
  \end{equation}
\end{proof}


%%% Exists partition from MB with maximum size %%%
% Status: OK
\begin{proposition}\label{prop:EMaxKappa}  
  Suppose $f:[a,b] \rightarrow \mathbb{R}$ is CPM. Then
  \begin{equation}\label{eq:MaxKappa}
    \exists \kappa_{ab} \in MB_{p}(f,[a,b]): \forall \kappa \in MB_{p}(f,[a,b]) [|\kappa_{ab}| \geq |\kappa|].
  \end{equation}  
\end{proposition}
\begin{proof}
  Suppose $f$ is CPM, then $\forall \kappa \in BP(f): |\kappa| \leq K(f)$.
  Since $MB_{p}(f) \subset BP(f)$, then $\forall \kappa \in MP_{p}(f): |\kappa|\leq K(f)$. 
  Therefore, the size of partitions from $MP_{p}(f)$ are bounded.
  As a result the size of the partition $|\kappa|$ is bounded natural number, 
  whereas such set has the larges element.
\end{proof}

%%% The partition of MB with maximum size (definition) %%%
% Status: OK
\begin{definition} \label{def:MB^m} 
  The set of partitions $\kappa_{ab}$ which satisfies \ref{eq:MaxKappa}
  will be denoted $MB_{p}^m(f,[a,b])=MB_{p}^m(f)$.  
  This set is non empty according to Proposition~\ref{prop:EMaxKappa}.

  In addition it could be shown (see Proposition~\ref{prop:E1Kappa}) 
  that the set $MB_{p}^m(f)$ has exactly one element and 
  \begin{equation}
    \forall \kappa \in MB_{p}(f):\kappa \subset \kappa_{ab} \in MB_{p}^m(f)
  \end{equation}
\end{definition}

%%% MP^m <=> p-variation => f-join %%%  
% Status: Need subtlety in proof
\begin{lemma}\label{lem:MinJunktSal} 
  Let $f$ is $CMD$. Then $\{a,b\} \in MB_{p}^m(f,[a,b])$
  if and only if
  \begin{equation*}
    \forall \kappa \in BP(f,(a,b)): v_p(f,[a,b]) > s_p(f, \kappa).
  \end{equation*} 
  Here the $BP(f,(a,b))$ is the set of partitions of break points
  excluding $\{a,b\}$ (see def.~\ref{eq:BP(f,(a,b))}).
\end{lemma}
\begin{proof} 
  Necessary. Let assume on the contrary that $\{a,b\} \in MP_{p}^m(f,[a,b])$, but
  \begin{equation*}
     \exists \kappa \in BP(f,(a,b)): v_p(f,[a,b]) = s_p(f, \kappa).
  \end{equation*} 
  
  The partition $\kappa \in BP(f,(a,b))$ must have more points then the end of the intervals $a$ and $b$,
  because if $\kappa = \{a,b\}$ then it contradicts to the definition of $BP(f,(a,b))$.
  Therefore, $|\kappa|>|\{a,b\}|$, but this contradicts to assumptions that $\{a,b\} \in MP_{p}^m(f,[a,b])$.
  
  Sufficiency. Suppose 
  \begin{equation}\label{eq:MinJunktSal_Ir_eq1}
    \forall \kappa \in BP(f,(a,b)): v_p(f,[a,b]) > s_p(f, \kappa).
  \end{equation} 
  
  According to Proposition~\ref{prop:pvar-max}, the $p$-variation is achieved in the set $BP(f,[a,b])$, 
  therefore, from the (\ref{eq:MinJunktSal_Ir_eq1}) follows that
  $\{a,b\}$ is the partition of meaningful break points, i.e. $\{a,b\} \in MP_{p}(f,[a,b])$. 
  In addition, this partition is the only one, therefore it is the biggest partition, thus 
  $\{a,b\} \in MP_{p}^m(f,[a,b])$.
\end{proof}    


%%% The subset of MB^m is MB^m %%%
% Status: OK
\begin{lemma}\label{prop:MaxKapaJunkMax}
  Let $f:[a,b] \rightarrow \mathbb{R}$ be CPM. Suppose $\{x_i\}_{i=0}^n \in MB_{p}^m(f,[a,b])$.
  Then 
  \begin{equation}
    \forall i \in \{1,2,\dots,n\} : \{x_{i-1},x_i\} \in MB_{p}^m(f,[x_{i-1},x_i]). 
  \end{equation}
\end{lemma}
\begin{proof}
  Suppose to the contrary that
  \begin{equation}
    \exists j \in \{1,2,\dots,n\} :\{x_{j-1},x_j\}  \notin MB_{p}^m(f,[x_{j-1},x_j]).
  \end{equation}
  
  Then, according to Lemma~\ref{lem:MinJunktSal} 
  \begin{equation}
    \exists \kappa \in BP(f,(x_{j-1}, x_j)): v_p(f,[x_{j-1}, x_j]) = s_p(f, \kappa).
  \end{equation}
    
  On the other hand, from the Proposition~\ref{prop:pvar_sum_x} we get
  \begin{eqnarray*}
    v_p(f,[x_0, x_n]) &=& v_p(f,[x_0, x_{j-1}]) + v_p(f,[x_{j-1}, x_{j}]) + v_p(f,[x_j, x_n])  \\
                &=& s_p(f,\{x_i\}_{i=0}^{j-1}) + s_p(f, \kappa) +s_p(f,\{x_i\}_{i=j}^{n}).
  \end{eqnarray*}
  Thus,  $\{x_i\}_{i=0}^{j-1} \cup \kappa \cup \{x_i\}_{i=j}^{n} \in MB_{p}(f,[a,b])$.  
  Therefore, $\{x_i\}_{i=0}^n \notin MB_{p}^m(f,[a,b])$, because
  \begin{equation}
  |\{x_i\}_{i=0}^{j-1} \cup \kappa \cup \{x_i\}_{i=j}^{n}| > |\{x_i\}_{i=0}^n|,  
  \end{equation}
  This contradicts the initial assumption.
\end{proof}

\subsection{f-join}

%%% $f$-join (definition) %%%
% Status: OK
\begin{definition}[$f$-join]\label{def:fjoin}
  Suppose $f:[a,b] \rightarrow \mathbb{R}$ is $CPM$.
  We will say that points $t_a$ and $t_b$ ($t_a<t_b$) are \emph{$f$-joined} in interval $[a,b]$ if 
  \begin{equation}
    \exists \{x_j\}_{j=0}^{n} \in MB_{p}(f,[a,b]): [t_a,t_b]=[x_{j-1},x_j], \text{ with some } j.
  \end{equation}  
\end{definition}



%%% {a,b} are f-foined => v_p([a,b]) = |f(a)-f(b)|^p %%%
% Status: OK
\begin{lemma}\label{prop:fjoinPvar} 
  Suppose $f:[a,b] \rightarrow \mathbb{R}$ is CPM and $a \leq t_a<t_b \leq b$.
  If points $t_a$ and $t_b$ are $f$-joined, then 
  \begin{equation}
    v_p(f,[t_a,t_b])=|f(t_b)-f(t_a)|^p. 
  \end{equation}
\end{lemma}
\begin{proof} 
  Let $f:[a,b] \rightarrow \mathbb{R}$ is CMD, 
  $t_a<t_b$ and points $t_a$ and $t_b$ are $f$-joined. 
  Then exists $\{x_j\}_{j=0}^{n}$ and $j$ from the Definition~\ref{def:fjoin}.
  According to definition of the $p$-variation and properties of $s_p$ function we get
  \begin{equation}
  v_p(f,[a,b])= s_p(f,\{x_i\}_{i=0}^{K(f)}) = 
     s_p(f,\{x_i\}_{i=0}^{j-1}) + s_p(f,\{x_{j-1},x_j\}) + s_p(f,\{x_i\}_{i=j}^{K(f)}).
  \end{equation}
  
  Suppose to the contrary that $\exists r \in PP_{p}(f,[x_{j-1},x_j]): s_p(f,r)>s_p(f,\{x_{j-1},x_j\})$, thus
  \begin{equation}
    v_p(f,[a,b]) < s_p(f,\{x_i\}_{i=0}^{j-1}) + s_p(f,r) + s_p(f,\{x_i\}_{i=j}^{K(f)}) = s_p(f,\kappa),
  \end{equation}
  where $\kappa:=\{x_i\}_{i=0}^{j-1} \cup r \cup \{x_i\}_{i=j}^{K(f)}$.
  The last inequality contradicts the definition of $p$-variation, 
  therefore $\{x_{j-1},x_j\}$ is the partition of meaningful break points in interval $[x_{j-1},x_j]$, thus
  \begin{equation}
    v_p(f,[x_{j-1},x_j])=s_p(f,\{x_{j-1},x_j\})=|f(x_j)-f(x_{j-1})|^p.
  \end{equation}    
\end{proof}


%%% The extremum of the function (definition) %%%
% Status: OK
\begin{definition}[Extremum]\label{def:extremum}
  We will call the point $t \in  [a,b]$ an \emph{extremum} of the function $f$ in interval $[a,b]$ if   
  $f(t)=\sup \left\{ f(z):z\in \left[ a,b\right]
  \right\} $ or $f(t)=\inf \left\{ f(z):z\in \left[ a,b\right] \right\} $.
\end{definition}

%%% Pseudo-monotonic function (definition) %%%
% Status: OK
\begin{definition}[Pseudo-monotonic function]\label{def:PsiaudoMon}
  A function $f$ is called \emph{pseudo-monotonic} in interval $[a,b]$ if
  \begin{equation}
    \forall x \in [a,b]: f(x) \in [\min(f(a),f(b)), \max(f(a),f(b))].  
  \end{equation}
\end{definition}

%%% f-joined points are extremum of inner interval %%%
% Status: OK
\begin{proposition}\label{prop:MinMax_in_fjoin}
  Let $f:[a,b] \rightarrow \mathbb{R}$ be CPM and points $t_a$ and $t_b$ are $f$-joined.
  Then points $t_a$ are $t_b$ extremums of the function in the interval $[t_a,t_b]$.
  In addition, there is no other break point $d \in BP(f, (t_a,t_b))$ that
  is an extremum of the function in the interval $[t_a,t_b]$.
\end{proposition}
\begin{proof}
  Let $f:[a,b] \rightarrow \mathbb{R}$ be CPM and points $t_a$ and $t_b$ are $f$-joined.
  Since $f \stackrel{PM}{=} -g$, with out loss of generality we can assume that
  $f(t_a) \leq f(t_b)$.  
  Suppose to the contrary that $f(t_b)$ 
  is not an extremum of the function in interval $[t_a,t_b]$.
  Hence, $\exists c \in [t_a, t_b]: f(c)>f(t_b)$. Therefore,
  $|f(c)-f(t_a)|^p>|f(t_b)-f(t_a)|^p$.
  According to Proposition \ref{prop:fjoinPvar}, 
  $v_p(f,[t_a,t_b])=|f(t_b)-f(t_a)|^p$, 
  thus, $|f(c)-f(t_a)|^p>v_p(f,[t_a,t_b])$,
  but this contradicts the definition of $p$-variation.  
  So, point $t_b$ must be an extremum in interval $[t_a,t_b]$.
  Completely symmetric arguments could be used for point $t_a$.
  
  In addition, it could be shown that if $d \in BP(f, (t_a,t_b))$, then
  $f(d) < f(t_b)$.  
  Suppose to the contrary that $\exists d \in BP(f, (t_a,t_b)): f(d) = f(t_b)$.
  Then, $|f(d)-f(t_a)|^p = |f(t_b)-f(t_a)|^p$, 
  therefore, $v_p(f,[t_a,d])=v_p(f,[t_a,t_b])$.
  Since, $v_p(f,[t_a,t_b]) \geq v_p(f,[t_a,d])+v_p(f,[d,t_b])$, 
  we get that $v_p(f,[d,t_b]) \leq 0$. 
  So, $v_p(f,[d,t_b]) = 0$, because $v_p(f,[d,t_b]) < 0$ is not valid.
  According to Lemma~\ref{lm:element_properties}(\ref{lm:ep_b}), $v_p(f,[d,t_b]) = 0$ iff function is constant in $[d,t_b]$.
  But this contradicts to the fact that $d$ is a break point (see Lemma~\ref{lem:noncons}).  
\end{proof}
\begin{corollary}\label{cor:pseudo-monotonic}
  Let $f:[a,b] \rightarrow \mathbb{R}$ be CPM and points $t_a$ and $t_b$ are $f$-joined.
  Then function is pseudo-monotonic in interval $[t_a,t_b]$. 
  So, if $f(t_a) \leq f(t_b)$, then
  \begin{equation}
    \forall x \in (t_a,t_b): f(t_a) \leq f(x) \leq f(t_b).
  \end{equation}
  In addition,  
  \begin{equation}
    \forall d \in BP(f,(a,b)):f(a)<f(d)<f(b) .
  \end{equation}
\end{corollary}
\begin{proof}
  The result follows directly from the Proposition~\ref{prop:MinMax_in_fjoin}. 
\end{proof}


%%% Global extremums are meaningful breaks %%% 
% Status: OK
\begin{proposition}\label{prop:SplitMinMax}
  Let $f:[a,b] \rightarrow \mathbb{R}$ is $CPM$. 
  If point $d \in [a,b]$ is a break point, which is an extremums of the function $f$, 
  then $d$ is a meaningful break point.
\end{proposition}
\begin{proof}
  Let $f:[a,b] \rightarrow \mathbb{R}$ is $CPM$.
  Suppose to the contrary that exists $d \in \overline{BP}(f,[a,b])$, 
  which is an extremum of the function, but
  $d \notin \overline{MB}_{p}(f,[a,b]) $.
  Let $\{x_j\}_{j=0}^{n} \in MB(f,[a,b]$ be any partition from $MB(f)$. 
  Since $d \in [a,b]$ and $d \notin \overline{MB}_{p}(f,[a,b])$,
  \begin{equation}
    \exists j \in 1,\dots,n: d \in (x_{j-1},x_j) .
  \end{equation}  
  The points $x_{j-1}$ and $x_j$ are $f$-joined and $d$ is break point, 
  hence, according Corollary~\ref{cor:pseudo-monotonic}
  \begin{equation}
    f(x_{j-1})<f(d)<f(x_j).
  \end{equation}
  But this contradicts the assumption that $d$ is the extremum.
\end{proof}


\subsection{Main result}

%%% The monotonisy of Sp function %%%
% Status: Need proof in caorollary
\begin{lemma}\label{prop:Sp_Monoton}
  Suppose $C \in \mathbb{R}$, $c_1\geq0$ and $1\leq p<\infty$. Then function
  $f:[0,\infty) \rightarrow \mathbb{R}$ with the values
  \begin{equation}
  f(x) = ( x + c_1  )^p - x^p - C,\;x \in [0,\infty),  
  \end{equation}
  are non decreasing in interval $[0,\infty)$. 
\end{lemma}
\begin{proof}
  Suppose $C \in \mathbb{R}$, $c_1\geq0$ and $1\leq p<\infty$. 
  The the derivative of the function $f$ is
  \begin{eqnarray*} 
    f'(x)   &=& p(x+c_1)^{p-1} - px^{p-1} \\
          &\geq& p(x)^{p-1} - px^{p-1} = 0.
  \end{eqnarray*}  
  The derivative of function $f$ is non negative, thus the function $f$ is non decreasing.
\end{proof}
\begin{corollary}\label{cor:convex}
  Suppose $c_1\geq0$, $1\leq p<\infty$ and $0 \leq x \leq y$. Then the following implication holds 
  \begin{equation}
  |x + c_1 |^p > x^p + C \Rightarrow |y + c_1 |^{p} > y^{p} + C.  
  \end{equation}
\end{corollary}
\begin{proof}
  Suppose  $0 \leq x \leq y$. 
  Since $f$ is non decreasing, $f(x) \leq f(y)$.
  Therefore, if $f(x)>0$, then $f(y)>0$.
\end{proof}


%%% VIP: Meaningless if in inner MP^m %%%
% Status OK
\begin{proposition}\label{prop:NotInMB}
  Let $f:[a,b] \rightarrow \mathbb{R}$ is CPM,
  $1 \leq p < \infty$, $[a',b'] \subset [a,b] $. If 
  \begin{equation}\label{eq:NotInMB1}
    \{a',b'\} \in MB_p^m(f,[a',b']),
  \end{equation}
  then
  \begin{equation}\label{eq:NotInMB2}
    \forall x \in (a',b'): x \notin \overline{MB}_p(f,[a,b])    
  \end{equation}
\end{proposition}
\begin{proof}
  Suppose to the contrary that the assumptions of the proposition is valid and    
  \begin{equation}\label{eq:NotInMB_p1}
    \exists x \in (a',b'): x \in \overline{MB}_p(f,[a,b]).
  \end{equation}
  
  According to the definition of $\overline{MB}_p(f,[a,b])$, 
  the (\ref{eq:NotInMB_p1}) means that
  \begin{equation}\label{eq:NotInMB_p3}
    \exists \{t_i\}_{i=0}^n \in MB_{p}(f,[a,b]): x \in \{t_i\}_{i=0}^n.
  \end{equation}  
  
  Moreover, 
  \begin{equation}\label{eq:NotInMB_p2} 
    x \notin \{a, a', b', b\},
  \end{equation} 
  because $x \in (a',b')$.  
  
  Thus, from (\ref{eq:NotInMB_p2}) and (\ref{eq:NotInMB_p3}) it follows that  
  \begin{equation}\label{eq:NotInMB_p4}
    \exists j \in \{1,2,\dots,n-1\} : x = t_j.
  \end{equation}  

  Lets denote variables $l$ and $k$ by
  \begin{eqnarray}
    l &:=& \max \left\{i \in \{1,2,\dots,n-1\} : t_i  \in (a',b') \right\}, \\
    k &:= &\min \left\{i \in \{1,2,\dots,n-1\} : t_i  \in (a',b') \right\}.
  \end{eqnarray}  
  Since (\ref{eq:NotInMB_p4})~holds, the values $l$ and $k$ always exists. 
  According to $l$ and $k$ definitions the following inequality holds  
  \begin{equation}\label{eq:NotInMB_p5}
    t_{k-1} \leq a' < t_{k} \leq t_{l} < b' \leq t_{l+1}.
  \end{equation}
  
  Points $a'$ and $b'$ are f-joined according to (\ref{eq:NotInMB1}), therefore,
  function $f$ is pseudo-monotonic in interval $[a',b']$. 
  Since $v_p(f)=v_p(-f)$ (see Proposition~\ref{prop:f_PM_g}), with out loss of generality 
  we can assume that interval $[a',b']$ is decreasing, i.e. 
  $$f(a') > f(b').$$  
  
  Moreover, points $t_l$ and $t_k$ are breakpoints, 
  because $\{t_i\}_{i=0}^n \in MB_{p}(f,[a,b]) \subset BP(f,[a,b])$.
  Therefore, according to Corollary~\ref{cor:pseudo-monotonic} 
  the values $f(t_{l})$ and $f(t_{k})$ can be only between $f(a')$ and $f(b')$. So,
  \begin{eqnarray}
    f(a') > f(t_{l}) > f(b'), \label{eq:NotInMB_pd1}  \\
    f(a') > f(t_{k}) > f(b'). \label{eq:NotInMB_pd2}
  \end{eqnarray}
  
  In addition, since the pairs of points $\{t_l,t_{l+1}\}$ and $\{t_k,t_{k+1}\}$ are also f-joined, 
  the inequalities (\ref{eq:NotInMB_pd1}) and (\ref{eq:NotInMB_pd2})
  could be extended as follows 
  \begin{eqnarray} 
      f(a') > f(t_{l}) > f(b') \geq f(t_{l+1}), \label{eq:NotInMB_p6.1}    \\
      f(t_{k-1}) \geq f(a') > f(t_{k}) > f(b'). \label{eq:NotInMB_p6.2} 
  \end{eqnarray}  
  
  Since $\{a',b'\} \in MB_p^m(f,[a',b'])$, by the Lemma~\ref{lem:MinJunktSal},
  any other partition in interval $[a',b']$ 
  is not a meaningful partition, thus 
  \begin{equation}
  |f(a')-f(b')|^p = v_p(f,[a',b']) > v_p(f,[a',t_{k}]) + v_p(f,[t_{k},t_{l}]) + v_p(f,[t_{l},b']). 
  \end{equation}
  According to Lemma~\ref{lm:element_properties}(\ref{lm:ep_g}), from the following proposition follows
  \begin{eqnarray*} 
    |f(a')-f(b')|^p &>& |f(a')-f(t_{k})|^p + v_p(f,[t_k,t_l]) +  |f(t_{l})-f(b')|^p \\
    |f(a')-f(t_l)+f(t_l)-f(b')|^p &>& |f(a')-f(t_{k})|^p + v_p(f,[t_k,t_l]) +  |f(t_{l})-f(b')|^p.  \label{eq:NotInMB_p7} 
  \end{eqnarray*}
  From the (\ref{eq:NotInMB_p6.1}) the following inequalities holds
  \begin{eqnarray*}
    f(a')-f(t_l) &>& 0, \\
    f(t_{l})-f(t_{l+1}) \geq f(t_l)-f(b') &>& 0.
  \end{eqnarray*} 
  Therefore, we can use Corollary~\ref{cor:convex}. According to it,
  from the inequality (\ref{eq:NotInMB_p7}) follows  
  \begin{eqnarray*}
    |f(a')-f(t_l)+f(t_l)-f(t_{l+1})|^p &>& |f(a')-f(t_{k})|^p + v_p(f,[t_k,t_l]) +  |f(t_{l})-f(t_{l+1})|^p \\
    |f(a')-f(t_{l+1})|^p &>& |f(a')-f(t_{k})|^p + v_p(f,[t_k,t_l]) +  v_p(f,[t_l,t_{l+1}]). 
  \end{eqnarray*} 
  Absolutely symmetric argument could be used in other direction. Firstly lets modify last inequality
  $$  |f(a')-f(t_k)+f(t_k)-f(t_{l+1})|^p > |f(a')-f(t_{k})|^p + v_p(f,[t_k,t_l]) +  v_p(f,[t_l,t_{l+1}]), $$
  then form the  (\ref{eq:NotInMB_p6.2}) holds the following inequalities
  \begin{eqnarray*}
    f(t_k)-f(t_{l+1}) &>& 0, \\
    f(t_{k-1})-f(t_{k}) \geq f(a')-f(t_k) &>& 0.
  \end{eqnarray*} 

  Thus, from Corollary~\ref{cor:convex} we get
  \begin{eqnarray*}
    |f(t_{k-1})-f(t_k)+f(t_k)-f(t_{l+1})|^p&>&|f(t_{k-1})-f(t_{k})|^p+v_p(f,[t_k,t_l])+v_p(f,[t_l,t_{l+1}]) \\
    |f(t_{k-1})-f(t_{l+1})|^p &>& v_p(f,[t_{k-1},t_{k}]) + v_p(f,[t_k,t_l])+v_p(f,[t_l,t_{l+1}])
  \end{eqnarray*}  
  Finlay, using Lemma~\ref{prop:pvarSubset} we conclude that
  $$|f(t_{k-1})-f(t_{l+1})|^p > v_p(f,[t_{k-1},t_{l+1}]).$$
  This contradicts with the definition of $p$-variation.  
\end{proof}


%%% VIP: In meaningless => meaningless %%%
% Status: OK
\begin{proposition}\label{prop:non_non}
  Suppose $f:[a,b] \rightarrow \mathbb{R}$ is CPM,
  $1 \leq p < \infty$ and $x \in [a,b]$. Then the proposition
  \begin{equation}\label{eq:NN1}
    T:=\exists (a',b') \in \left\{ (u,v) \in \mathbb{R}^2: x \in [u,v] \subset [a,b]\right\}: x \notin \overline{MB}_p(f,[a',b'])
  \end{equation}
  is equivalent to
  \begin{equation}\label{eq:NN2}
   S:= x \not\in \overline{MB}_p(f,[a,b]).
  \end{equation}
\end{proposition}
\begin{proof}
  Sufficiency. Suppose $ x \notin \overline{MB}_p(f,[a,b])$. 
  Then the proposition (\ref{eq:NN1}) is obviously true then
  $[a',b'] = [a,b]$.
  
  Necessity.
  Suppose $f:[a,b] \rightarrow \mathbb{R}$ is CPM,
  $1 \leq p < \infty$, $x \in [a,b]$ and (\ref{eq:NN1}) holds.

  
  Let $\{t_i\}_{i=0}^n \in MB_{p}^m(f,[a',b'])$. 
  Since $x \in [a',b']$, then $\exists j: x \in (t_{j-1},t_j)$. 
  In addition, the Lemma~\ref{prop:MaxKapaJunkMax} ensures that $\{t_{j-1},t_j\} \in MB_{p}^m(f,[t_{j-1},t_j])$.
  Thus, we can apply Proposition~\ref{prop:NotInMB} for interval $[a,b]$ and point $x$. 
  According to it $x \notin \overline{MB}_p(f,[a,b])$.
\end{proof}
\begin{corollary}\label{cor:non_non}
  Since $(T \Leftrightarrow S) \Leftrightarrow (\neg T \Leftrightarrow \neg S) $, we get
  $$ x \in \overline{MB}_p(f,[a,b]) \Leftrightarrow \forall [a',b'] \subset [a,b], x \in [a',b']: x \in \overline{MB}_p(f,[a',b']). $$
\end{corollary}

\subsection{Extra conclusions}

%%% The uniqueness of partition with maximum size %%%
% Status: OK
\begin{proposition}\label{prop:E1Kappa}
  Suppose $f:[a,b] \rightarrow \mathbb{R}$ is CPM,
  $1 \leq p < \infty$. Let $\kappa_{ab} \in MP_{p}^m(f,[a,b])$, then
  \begin{equation}
  x \in \overline{MB}_{p}(f,[a,b]) \Leftrightarrow  x \in \kappa_{ab}.
  \end{equation}
\end{proposition}
\begin{proof}
  Sufficiency. If $x \in \kappa_{ab} \in MP_{p}^m(f,[a,b])$, then
  $x \in \overline{MB}_{p}(f,[a,b])$ according to the definition of  
  $\overline{MB}_{p}(f,[a,b])$ (see~Definition~\ref{def:MB}).
  
  Necessary. Suppose to the contrary that $x \in \overline{MB}(f,[a,b])$, 
  but $x \not \in \kappa_{ab} \in MB_{p}^m(f,[a,b])$.
  The points of the partition $\kappa_{ab}$ will be denoted $t_i$, i.e. $\kappa_{ab}=\{t_i\}_{i=0}^n$.
  Then
  $$ \exists j\in\{1,2,\dots,n\}:x \in (t_{j-1},t_j).$$
  Since $\{(t_{j-1},t_j\} \in MB_p^m(f,[t_{j-1},t_j])$,  $x \notin \overline{MB}_p(f,[t_{j-1},t_j])$.
  Therefore, we can apply Proposition~\ref{prop:non_non}. From it follows that  
  $$ x \notin \overline{MB}_{p}(f,[a,b]).$$
  This is the contradiction that proves the proposition.
\end{proof}
\begin{corollary} 
  According to the definition of MB (see~Definition~\ref{def:MB}),
  from the last proposition directly follows that  
  $$\forall \kappa \in MB_{p}(f,[a,b]): \kappa \subset \kappa_{ab}, $$
  this means
  $$\exists!\kappa_{ab} \in MB_{p}^m(f,[a,b]). $$
\end{corollary} 


%%% Monotonicity of p-variation sum function %%%
% Status: ???
\begin{lemma}\label{prop:Pmonoton}
  Let $C\geq0$ and $c_i\geq0,\;i=1,\dots,n$. Suppose $1 \leq p < q < \infty$. Then the implication holds
  $$C^p > \sum_{i=1}^n c_i^p  \Rightarrow C^q > \sum_{i=1}^n c_i^q. $$ 
\end{lemma}
\begin{proof}
  Let denote the function $f(p)$ as 
  $$f(p) = C^p - \left[ \sum_{i=1}^n c_i^p \right].$$
  The derivative of function $f$ is
  \begin{eqnarray*}
    f'(p) &=& C^p\log{p} - \left[ \sum_{i=1}^n c_i^p \log{p} \right] \\
            &=& \left(C^p - \left[ \sum_{i=1}^n c_i^p \right]\right)\log{p} .
  \end{eqnarray*}
  Thus, if $1 \leq p < \infty$ and $C^p > \sum_{i=1}^n c_i^p$, then 
  $f$ derivative is non negative. This means that function $f$ is non decreasing, 
  whereas this proof the Lemma~\ref{prop:Pmonoton}
\end{proof}

%%% The monotonicity of partition according to p %%%
% Status: OK
\begin{proposition}\label{prop:MBMonoton} 
  Suppose $f:[a,b] \rightarrow \mathbb{R}$ is CPM and 
  $1 \leq p < q < \infty$. Then
  $$\overline{MB}_{q}(f,[a,b]) \subset \overline{MB}_{p}(f,[a,b]). $$
\end{proposition}
\begin{proof}
  Suppose to the contrary that
  \begin{equation*}
  \exists x \in[a,b]:x \in \overline{MB}_{q}(f,[a,b]),\,x \notin \overline{MB}_{p}(f,[a,b]).   
  \end{equation*}

  Let $\{t_i\}_{i=0}^n \in MB_p^m(f,[a,b])$. 
  Because $x \in [a,b]$ and $x \notin \overline{MB}_{p}(f,[a,b])$, then
  \begin{equation}\label{eq:xinj}
    \exists j \in\{1,2,\dots,n\}:x\in (t_{j-1},t_j). 
  \end{equation}
  Let arbitrary choose $\kappa \in BP(f,(a,b))$.
  Then according to Lemma~\ref{lem:MinJunktSal}, the following inequality holds
  \begin{equation*}
    |f(t_{j-1})-f(t_{j})|^p > s_p(f, \kappa).
  \end{equation*}
  Whereas, according to Lemma~\ref{prop:Pmonoton},
  \begin{equation*}
    |f(t_{j-1})-f(t_{j})|^q > s_q(f, \kappa).
  \end{equation*} 
  This means that
  $$\{t_{j-1}, t_j \} \in MB_q^m(f,[t_{j-1},t_j]).$$ 
  Finlay, from the Proposition~ \ref{prop:NotInMB} follows that $x \notin \overline{MB}_q(f,[a,b])$,
  but this contradicts the assumptions.
\end{proof}
  
\section{p-variation calculus}  
  
This chapter will present the algorithm that calculates 
the $p$-variation for arbitrary piecewise monotone function. 
This algorithm will be called \emph{pvar}. 
It is already realised in the R (see ???) package \emph{pvar} and
is publicly available on CRAN.

Let assume $f$ is any piecewise monotone function.  
According to definition (see~\ref{def:PM}) 
there are points $a=x_1<\dots<x_n=b$ for some finite $n$ such 
that $f$ is monotone on each interval $[x_{j-1},x_j],\;j=1,2,\dots,n$.
The sequence of values of the function $f$ 
at the points of partition $\{x_{i}\}_{i=0}^{n}$ will be
referred as \emph{sample} and denoted $X=\{X_{i}\}_{i=0}^{n}:=\{f(x_{i})\}_{i=0}^{n}$. 
The value of $n$ will be called \emph{sample size}.


The algorithm \emph{pvar} do not use $f$ directly, 
rather it uses the sample $X$ as an input 
and operates it to find the $p$-variation.

\subsection{Algorithm}

The procedure \emph{pvar} that calculates
$p$-variation will be presented here. Firstly,
we will introduce the main schema, further, each step will
be discussed in more details.

\paragraph{Procedure \emph{pvar}.}
Input: sample $X$, scalar $p$.

\begin{enumerate}
  \item \emph{Removing monotonic points}. According to proposition ???
  all points that are not the end of monotonic interval could be excluded 
  from further consideration. 
  
  \item \emph{Checking all small intervals}. Every sm

  \item \emph{Partition spliting}. 
  
  \item \emph{Indipenden intervals assesments}.
  
  \item \emph{Joining results}.  
  
\end{enumerate}

Pseudo-code
\begin{lstlisting}
pvar <- function(x, p){
  xNew = Remove_Monotonic_Points(x) 
  xNew = Test_Points_In_Small_Intervals(xNew, p, sizeN = 7) 
  IntervalList = Split_By_Extremums_All(xNew)
  for(Interval in IntervalList){
    Partition = Join(Partition, pvarMon(Interval, p, sizeN))
  }
  pvar = Pvar_Sum(Partition, p)  
  return(pvar)
}
\end{lstlisting}
  


\section{p-variation for pseudo-monotonic sample}  

Let suppose that the sample is
\begin{itemize}
  \item Pseudo monotonic.
  \item There are no monotonic points.
  \item All possible small intervals are checked.
\end{itemize}


\begin{enumerate}
  \item Check if the given interval is short.



\end{enumerate}

Pseudo kodas:
\begin{lstlisting}
pvarMon <- function(Interval, p, sizeN){
  # Jei tai trumpas intervalas, tai jis jau buvo patikrintas 
  if(length(Interval)<=sizeN){
    return(Interval) # tiesiog grazinamas intervalas  
  }

  # Kitu atveju skaiciuojam giliau
  NewIntervalList = Split_By_CumExtremums(x[a:b])
  for(Int in NewIntervalList){
    (int1, int2) = Split_MinMax(Int)
    Partition1 = pvarMon(int1)
    Partition2 = pvarMon(int2)
    PassiblePartition = Join(PassiblePartition, Partition1, Partition2)
  }

  # Patikriname pontecialius skaidinio tasku,
  # is eiles tikrindami galimas jungtis
  PointList = EndPoints(NewIntervalList)
  for(Point in PointList){
    SubInt = Subset(PassiblePartition, from=1, to=Point)
    if(Pvar_Sum(SubInt, p)<Pvar_Sum({1,Point}, p)){
      DropPoints = SetDiff(PassiblePartition, {1,Point})
      PassiblePartition = SetDiff(PassiblePartition, DropPoints)
    }  
  }  
  return(PassiblePartition) 
}
\end{lstlisting}
  
  
%%%%%%%%%%%%%%%%%%%%%%%%%%%%%%%%%%%%%%%%%%%%%%%%%%%%%%%%%%%%%%%%%%%%%%%%%%%%%%%%%%%%%%%%%%% 
\begin{thebibliography}{99}  
  \bibitem{Qian} J. Qian. The $p$-variation of Partial Sum Processes
  and the Empirical Process // Ph.D. thesis, Tufts University, 1997.
\end{thebibliography}

 




\end{document}
